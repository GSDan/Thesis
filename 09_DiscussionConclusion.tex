\chapter{Discussion and Conclusion}

OurPlace can be used for to create the same type of worksheet style tasks that blumenfeld disliked, as well as PBL. Community members had also recognised this (check that quote was included) 

Previous work by Chatting et al. \citep{Chatting2017} has considered how an individual’s agency (argued to be a component of empowerment \citep{Ibrahim2007}) can be supported in the new Internet of Things world by applying lessons learned from the DIY movement to new, digital technologies. We believe that parallels can be drawn to this with our teachers and park volunteers: they were able to fulfil their goal through technology, (mostly) independent of the usual top-down institutional restrictions which would have affected their creative control and output (DG1). Uphoff argues that an empowerment process needs to provide access to ‘power resources’—the assets which create possibilities for achieving objectives [38]. For our teachers and volunteers, Activity authorship was a power resource: it allowed them to create their own content as they saw fit and release it in their own timeframe, with minimal top-down assistance. We suggest that future m-learning designs should consider how they can empower the user through content ownership. In ParkLearn, this was achieved by granting more creative control to users and elevating them from consumers to producers of educational content.

\section{Overview}

\section{Recommendations for Utilising Public Places as Infrastructure for Civic Mobile Learning}

\subsection{Genuine and Meaningful Engagement}
\subsection{Support Requirements}
\subsection{Providing Sustainable Solutions}