\chapter{Discussion and Conclusion}

Through a number of 

OurPlace can be used for to create the same type of worksheet style tasks that blumenfeld disliked, as well as PBL. Community members had also recognised this (check that quote was included) 

Previous work by Chatting et al. \citep{Chatting2017} has considered how an individual’s agency (argued to be a component of empowerment \citep{Ibrahim2007}) can be supported in the new Internet of Things world by applying lessons learned from the DIY movement to new, digital technologies. We believe that parallels can be drawn to this with our teachers and park volunteers: they were able to fulfil their goal through technology, (mostly) independent of the usual top-down institutional restrictions which would have affected their creative control and output (DG1). Uphoff argues that an empowerment process needs to provide access to ‘power resources’—the assets which create possibilities for achieving objectives [38]. For our teachers and volunteers, Activity authorship was a power resource: it allowed them to create their own content as they saw fit and release it in their own timeframe, with minimal top-down assistance. We suggest that future m-learning designs should consider how they can empower the user through content ownership. In ParkLearn, this was achieved by granting more creative control to users and elevating them from consumers to producers of educational content.

\section{Supporting Place-Making with Civic Mobile Learning Technologies}

Relph - the majority of modern experiences of landscapes are mediated by machines

`What begins as undifferentiated space becomes place as we get to know it better and endow it with value'. - Tuan

Thus, place is a social, spatio-temporal value, where space and time go together in shaping a person’s interpretations. People who inhabit the same physical space may, due to differing past experiences, associate the space with different meanings and values.
 
sharing these experiences with newcomers can make their understanding of the space as a place less abstract, and help them understand what makes a place special. 

volunteers in heritage workshop - However, Relph claims that this is increasingly being overshadowed in our modern era by an attitude he refers to as‘placelessness’:‘the casual eradication of distinctive places and the making of standardised landscapes that results from an insensitivity to the significance of place’

the creation of mobile learning activities can give opportunities for highlighting and sharing place attachment (the degree to how much someone values or identifies with a place, from it from fulfilling their needs or defining them as an individual) and place meaning (the meanings that individuals ascribe to settings that they are familiar with, reflecting their environment, social interactions, culture,politics, economics and history) (as seen in CrowdMemo)

- place-making occurs through experiencing place over time - mobile learning technologies offer new opportunities to experience place
    -novel interactions can be enabled between local communities and their places of heritage within authentic environmental settings. - (cross-media interaction) Giaccardi
    - in authentic learning contexts (encouraging a genuine engagement with the environment andstimulated pause and reflection - crivellaro), or
    - remotely, through vicarious insideness (comparison to Google Earth VR, but possibility for responding with own spatial context)
    - Invitation from a stakeholder to join `inside' a place by becoming more familiar with it
    - mobile devices are intrinsically personal - McCarthy and Wright 
    - allow for explorations of private interpretations or subversions of place within public space (RIOT!1831, Google Earth, Ged plaques)


the importance of the individual stakeholder sharing their point of view - Relph argues that this could partly be due to how space is perceived at a larger scale–he posits that the identity of a place becomes less authentic the more people share that identity.  He gives the example:‘a city presents a different image to those living in its mansions and those living in its slums’. The ‘public’ identity of the city would involve a consensus understanding of its image, representing the outlooks of it by all communities. The result is something superficial and contrived, likely representing neither of these communities authentically.

McCarthy and Wright suggest that for technologies to help people feel ‘in place’, they should engage at a personal level, rather than treat them as an anonymous entity

supports heritage becoming a living practice (giving people active andsupportive roles, [engaging] them in connecting to each others’ experiences, considering eachother’s interpretations, and building insights that may lead to new meanings and relationships... express their perceptions, interpretations and expectations about the heritage - Giaccardi)

crivellaro - (anticipatory archaeology’)

dangers of place being misrepresented by those who are invested in its success or blind to its flaws -  Relph also warns against the dangers of ‘museumisation’—the simplification and sanitisation of history to create a more palatable ideal.  He argues that by highlighting only the best bits of local history, we run the risk of creating a ‘Disneyfied’,inauthentic image of place.
this is best combated by opening up the ability to create and share to all stakeholders

however, crivellaro study - the residents wanted to posit a realistic view which maintained a sense of optimism and used the negative associations they had with their homes more constructively

\section{Recommendations for Utilising Public Places as Infrastructure for Civic Mobile Learning}

\subsection{Genuine and Meaningful Engagement}

Relph asserts that by undervaluing places’ distinctive characteristics, being overly concerned with efficiency and accepting environments which are interchangeable to the point of anonymity, we run the risk of normalising‘inauthentic’experiences of place.

fox and le dantec - involve and emphasize the agency and perspectiveof community members from the outset, as they are the ones best positioned to inform thedesign process

\subsection{Support Stakeholder Requirements}


\subsection{Providing Sustainable Solutions}
working with and within stakeholder groups
balestrini note that a sense of ownership of the project wouldnot have been enough to sustain engagement with it—they argue that this was achieved byproviding value to all of the involved stakeholders (i.e. valuing the elderly participants’ lifeexperiences, giving the young students new technology skills, supporting the teachers inrunning an innovative educational project)
 facilitating a range of face-to-face social encounters can lead to discussion andongoing engagement
 Fox and Le Dantex - Community Historians - configure the project to be clearly and immediately advantageous to the community members, as collaborators rather than participants

\section{Limitations}

education
A limitation of this research is that they were only loosely tied into each school curriculum, and didn't assess learning outcomes. We hope that future engagements can offer deeper integration into existing curricula and feature formal assessments, supporting further insights into how PBML can fit into wider school systems. 

place stakeholders
To better understand these factors, it’s likely that we as researchers will have to growrelationships with a place over time through direct habitation, or at least involve existingstakeholders as participants who have ongoing relationships with it. However, this becomesboth more interesting and potentially problematic when accounting for Tuan’s arguments ofthe influence of personal experience–because a person’s place attachment is formed basedupon individual experience, researchers should be aware that their perceptions of place may not align with others’, running the risk of inappropriate design decisions if not handled with care

star vs workshop participants
Star recalls that few participants in one of her projects utilisedthe final system that her team designed, despite the researchers following the principlesof participatory design throughout the process.  They identified that this was not becauseof usability issues with the interface, but rather how their design was a poor fit with theinfrastructures the participants had to work with. 

digital civics technologies used to support automation/austerity measures

\section{Conclusion}