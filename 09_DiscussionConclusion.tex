\chapter{Discussion and Conclusion}

\section{Overview}

These studies provided us with discussion points which we believe should bear consideration in future designs:

\section{OurPlace: Meeting the Design Goals}

\subsection{Supporting Seamless Learning Practices}

The features and open nature of the application’s authorship process and website
component means that it arguably supports all ten of Wong and Looi’s dimensions
of mobile-assisted seamless learning [43].  This includes the four research and
design gaps which they identified: use of multiple device types in different
contexts (e.g. tablets and projector in the field and classroom), switching
between multiple learning tasks (e.g. through combining Action Types, promoting
different interactions and considerations on the part of the learner), knowledge
synthesis (e.g. potential for children to create peer-learning Activities based
around their own independent or group research) and the encompassing of multiple
pedagogical or learning activity models (e.g. moving from individual work with
tablets in an authentic context to collaborative classroom discussion around the
uploaded responses). ParkLearn fulfilled DG2 by incorporating these dimensions
of seamless learning, allowing it to be flexible enough for teachers to
incorporate different devices, contexts and pedagogical approaches into their
activities as they see fit.
  
Over the course of the study, the role of the application changed from being the
learning objective to becoming a teaching support tool. In the Year 2 class’s
first Activity, the technology took centre stage and became the learning focus.
This overbearing design meant that not only did the children have little agency
in their output, but they weren’t paying much attention to the learning
environment. As suggested by Richardson et al., mobile learning design should
aim to strike a balance between direct and technology-mediated environmental
interactions if it is to take advantage of that environment as a learning
resource [29]. The teacher’s later Activity designs sought to strike that
balance, preferring the Action Types which focussed on the learning environment
(Figure 3). The technology’s ‘novelty’ diminishing over time (a motivation for
having the study taking place over several months [32]) also led to fewer
distractions from the environment. The hundreds of photos, videos and audio
recordings created and uploaded by the children during their trips (Table 2)
suggest that the children were easily able to use the application to support
their creative output, implying that it was successful at implementing DG5 and
DG4.

By supporting the offline caching of teachers’ Activities and children’s
responses on devices shared between several students, the application supported
structured outdoor activities without the need for Internet access or a
one-to-one device-student ratio (DG6). The technology also helped the Year 2
teacher utilise the children’s existing work for new educational activities in
the classroom: using the ParkLearn website on her laptop and classroom
smartboard projector facilitated full class discussion of students’ work
uploaded from the tablets. Presenting the students’ responses on the website in
a similar format to how they’re displayed in the application (complete with the
teacher’s prompts, images and the app’s iconography) had two main advantages: it
allowed the teacher to review the children’s work in the context in which it was
first presented to them, and it also gave the students a familiar reference
point to support them in doing related work in a different environmental
context. Land’s argument that the use of visual elements can allow users of
varying abilities to partake in mobile learning activities [21] suggests that
young children would have struggled with the equivalent text-based, CSV style
table interface on WildKnowledge [42]. Through simple interfaces which ground
the learner’s context (DG4), ParkLearn supported transitioning between devices,
learning environments and related activities (DG2).

\subsection{Engagement and Empowerment Through Ownership}

Throughout the study, the students, teachers and volunteers valued having
ownership of their work. For example, the children took pride in their creations
and showed off them to anyone that would listen. They recaptured videos if their
narration could be improved; they deleted and reshot photographs if the framing
wasn’t up to their own standards. As noted by Teacher 2, this pride was evident
on return to the classroom where they were eager to revisit their creations.
However, this enthusiasm wasn’t there for viewing other children’s responses to
the same Activity. The teacher believed that ownership of the task was an
important contributing factor to the children’s enthusiasm. Her plans for
ParkLearn activities with her next class would involve groups all researching
different topics: she argued that this unique knowledge would lead to the
children becoming experts on their given subject, with the ownership of the task
and knowledge empowering them through the ability to teach their peers. A
natural progression of this would be for children to create ParkLearn activities
for each other, moving towards giving the students greater control and
supporting deeper reflection through content construction [14,17]. Success of
this approach can be seen in Mobilogue, where students’ ownership of their
created quizzes prompted greater engagement [16]. By supporting such different
lesson structures ParkLearn successfully implements DG3.

Previous work has considered how an individual’s agency (a component of
empowerment [18]) can be supported in the new Internet of Things world by
applying lessons learned from the DIY movement to new, digital technologies [7].
We believe that parallels can be drawn to this with our teachers and park
volunteers: they were able to fulfil their goal through technology, (mostly)
independent of the usual top-down institutional restrictions which would have
affected their creative control and output (DG1). Uphoff argues that an
empowerment process needs to provide access to ‘power resources’—the assets
which create possibilities for achieving objectives [38]. For our teachers and
volunteers, Activity authorship was a power resource: it allowed them to create
their own content as they saw fit and release it in their own timeframe, with
minimal top-down assistance. We suggest that future m-learning designs should
consider how they can empower the user through content ownership. In ParkLearn,
this was achieved by granting more creative control to users and elevating them
from consumers to producers of educational content.

\subsection{Supporting Civic Engagement and Inquiry}

The technology acted as a medium which facilitated civic participation, showing
an opportunity for m-learning technologies to act as ‘gateways’ to active
engagement with civic space or communities. This supports Richardson et al.’s
suggestion that m-learning can engage with spaces’ social infrastructures as
resources for civic learning [29]. Teacher 2 argued that an opportunity existed
for technology to highlight to the parents the value of the community resources
and the children’s impact on them as active stakeholders. As shown in the
example of the care home visit, this highlighting could also be used to learn
about the lives of members within communities who have been ostracised,
forgotten or underappreciated. Through supporting multimedia data collection and
sharing through multiple device types, seamless m-learning technologies can
facilitate the sharing of civic knowledge and values with a wider community.
While ‘Earwig’ had been impractical due to the lengthy upload process,
ParkLearn’s immediacy could support such interactions without disrupting
teachers’ workflow. Teacher 2 also noted that beyond simply including the
children’s parents, this could also be extended to sharing values and practices
in cross-cultural learning engagements (DG1). As previous work has shown,
multimedia data collected through mobile devices can be used as effective
cross-cultural learning resources [31]. However, opportunities exist to explore
how m-learning technologies can support civic inquiry. When combined with the
nQuire-it platform, Sense-it supported ‘citizen inquiry learning’ by acting as a
scientific toolkit [33]. We propose that mobile technologies could also act as
toolkits to support ‘civic inquiry learning’: fostering cross-cultural
communities of inquiry, through the design of creative learning activities to
share and enquire about civic values and practices.

\section{Recommendations for Utilising Public Places as Infrastructure for Civic Mobile Learning}

\subsection{Genuine and Meaningful Engagement}
\subsection{Support Requirements}
\subsection{Providing Sustainable Solutions}

\section{Limitations and Going Forward}

This study was partially limited by the time limitations placed upon our
participating teachers. The application did not see as much usage by the Year 6
class due to a more demanding curriculum (resulting in fewer field trips) and
the beginning of their exam season. Future work will further investigate the
app’s use with this age group. Additionally, the installation of the ‘talking
statue’ coincided with the end of the school term, meaning that we were unable
to use it as a learning resource with the school during this study. Accessing
community expertise through technology was something Teacher 2 claimed to have
not considered before, but said it was “something we would use and we would
access.” Future work will endeavour to investigate how community generated
mobile learning resources can be used in formal education contexts. The
generalizability of these studies may also be somewhat limited by the
application’s park branding and imagery: the Year 6 teacher only used the
application for the outdoor section of his class’s trip, opting to stow the
tablets away for their indoor explorations of the museum. Similarly, it took
several months for the Year 2 teacher to use the application in an activity
which didn’t relate to parks, plants or animals. Future work could expand on
these findings by investigating in other contexts with a context-neutral
branding, which may counteract this issue.

\section{Summary}
We have presented ParkLearn—a mobile learning platform designed to support
teachers and communities in creating, sharing and completing bespoke mobile
learning activities. ParkLearn facilitated mobile learning in a formal education
context as seamless support tool, flexible enough to support teachers in
designing activities across different devices and learning contexts. Simplified
processes and interfaces meant that uploading the children’s work easily fit
into the teachers’ workflow, promoting follow-up classroom activities and even
sharing the content in engagements between the school and the surrounding
community. Through supporting creativity and independence, the platform promoted
ownership of content, increasing learners’ engagement in follow-up activities.
This element of independence also allowed community experts to elevate
themselves to producers of rich, digital educational content—supporting them in
sharing their knowledge and values with a wider community by removing the
technical and financial barriers previously in place. We also identified
opportunities for HCI to support cross-cultural civic inquiry, encouraging
learners to share their values, knowledge and questions in a manner already
embraced by citizen science research.