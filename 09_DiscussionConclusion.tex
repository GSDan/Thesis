\chapter{Discussion and Conclusion}

This research has taken place against a backdrop of political and financial uncertainty for much of the UK. The two main demographics that the project engaged with---volunteer-driven heritage preservation groups, and teachers within the formal education system---have both been impacted in recent years by pressures imposed by top-down institutional policy, and funding cuts made to cope with austerity measures. In response, these groups have started to look towards more sustainable methods of utilising existing resources and novel methods of engagement. Mobile technologies clearly have a potential role in this space: their growing ubiquity in society, as well as the popularity of location-aware applications such as \textit{Pok\'emon Go}, have meant that a large number of community groups we talked to were actively seeking to be represented through mobile applications. Similarly, the utilisation of mobile technologies in schools has gained popularity---both for data collection on school trips, and in the classroom for research and content creation.

This project has aimed to explore this context as a design space for mobile learning technologies which harness places---and the communities that care for them---as resources for both learning within the formal education system, and informal knowledge sharing within wider communities. Furthermore, I wanted to explore how such technologies could be used by place stakeholders to further their groups' interests and agendas, and how such mediums could be used to share their outlooks and values with new audiences.

With these contexts and aims as a backdrop, this chapter will discuss the findings that have been identified during the previously discussed studies. These discussions will pertain to how mobile learning technologies can be configured to support place-making and recommendations as to how researchers and designers can better utilise the infrastructures of place as resources for civic mobile learning. I also discuss some of the study's limitations, before concluding by responding to the research questions laid out at the project's inception.


\section{Supporting Place-Making with Mobile Learning Technologies}

Our relationships to place are molded by the experiences and familiarity we have with space. As Tuan posited: `\textit{What begins as undifferentiated space becomes place as we get to know it better and endow it with value}' \citep{Tuan1978}. Similarly, Relph argues that the process of building a relationship with space involves encountering and having experiences with it \citep{Relph1976}. Furthermore, he notes that the majority of the experiences that people had with the landscapes around them in the 1970s were mediated by machines---he noted that while it is easy to view this as a factor which acted as a barrier separating people from authentically experiencing place, technologies such as cars opened up new opportunities for people to experience spaces that had not previously been accessible to them. Today, it's pretty clear that mobile technologies play the same role---simultaneously erecting barriers to distract people from authentic experiences in place, and opening up new opportunities to encounter places which would otherwise be too remote or abstract to be easily accessible. While we must be aware and wary of the former, the latter presents new and exciting opportunities for using technology to support place-making processes. For, unlike the automobile, digital technologies allow users to traverse more than just physical distances: they can be configured to support encountering different interpretations of place, which may not have been previously accessible (or visible) regardless of physical proximity.

As, according to Tuan, our interpretations of place are based upon our experiences with it, people who inhabit the same physical space may, due to differing past experiences, associate the space with different meanings and values. Giving stakeholders opportunities to share these experiences with others can make their understanding of the space as a place less abstract, and help them understand what makes a place special. I argue that these studies have demonstrated that mobile learning technologies such as OurPlace can be used as platforms to offer these opportunities.  These studies showed how the creation of mobile learning activities can give opportunities for highlighting and sharing what Kudryavtsev labelled \textit{place attachment} and \textit{place meaning} \citep{Kudryavtsev2012}. Place attachment---the degree to how much someone values or identifies with a place, from it fulfilling their needs or defining them as an individual---was demonstrated through the use of the app by park volunteers in Chapter \ref{chap:Community} to highlight their group's efforts and attempt to recruit new members. Meanwhile, place meaning---the meanings that individuals ascribe to settings that they are familiar with, reflecting their environment, social interactions, culture, politics, economics and history---was seen through the use of the app to discuss notable figures in local history in Chapter \ref{chap:Teachers}, or by the Showchildren in Chapter \ref{chap:student-created} to introduce their ways of life). This is not unique to OurPlace: another example is Balestrini's CrowdMemo project, during which a mobile technology acted as a platform for community storytelling \citep{Balestrini2014}.

This use of mobile technologies for highlighting place attachment and meaning could be particularly useful when used in places which are `under appreciated'. For example, during the ParkLearn workshop covered in Chapter \ref{chap:Community}, volunteers referred to the value they held in the imperfections (such as generations old graffiti) of the places they cared for. In some cases, making these safe or suitable for physical public access could sanitise them, eradicating what made the places special to the stakeholders in the first place. In these instances, mobile learning technologies offer a potential solution: allowing people to remotely experience place they cannot physically access, supporting the building of vicarious insideness and support for preservation. While such place-making interactions can also be supported by non-mobile technologies (for example, the use of Google Earth VR by Jeff Gerstmann discussed in Section \ref{sec:TechMediator}), the use of mobile learning technologies can also support the ability for remote place `visitors' to respond to activities through the use of their own space/place context (e.g. comparing and contrasting the learner's lived experiences with the place they are remotely encountering). More subversively, the grassroots nature of OurPlace's Activity creation process could also allow unofficial support for people entering these areas unsanctioned (something which the participants reported happened anyway). The ability to remotely open up these places without the need to sanitise their value could help counter Relph's concerns regarding `placelessness': `\textit{the casual eradication of distinctive places and the making of standardised landscapes that results from an insensitivity to the significance of place}' \citep{Relph1976}.

A key advantage of mobile learning technologies is that they allow for users to engage in authentic learning contexts---in both the humanist geographer sense that they can provide `genuine experiences' through unmediated access to place's social qualities and constructs \citep{Relph1976}, and also in the learning sciences sense, where Situated Learning Theory posits that legitimate peripheral participation in communities of practice \citep{lave1991situated}. Through supporting communities of practice in creating and sharing learning resources, mobile learning technologies make it easier for newcomers to engage in peripheral participation with those communities. Emphasising this link between the learning resources and the individuals/communities which create them further encourages the place-making process: learners are exposed to others' place attachment and place meaning, while also forging their own experiences in place via the technology medium. This exposure to communities of practice may not be immediately obvious to the learner---for example, the rangers and volunteers in Chapter \ref{chap:DesignSpace} aimed to nurture an appreciation of their place within the schoolchildren, rather than recruit them as volunteers. Sometimes this may be more explicit, as in the case of the `Talking Statue' project, where the OurPlace Activity included Information Tasks which described the volunteer group's work and how the user could get involved. However, even this was obscured behind the Activity's primary goal of delivering historical information about the park's heritage.

The combination of mobile hardware, wireless networking and easily configurable software enables these communities of practice to design and create novel interactions for use by others within an authentic learning environment. In this regard, OurPlace Activities could be classed within the scope of `cross-media interaction', as posited by Giaccardi et al \citep{Giaccardi2008}. They argue that the use of multiple forms of media and technology can create new forms of socio-technical infrastructure, allowing place-making through new cultural experiences and the exploration of people's relationship with place. Giaccardi et al. also argue the importance of making heritage a `living practice' through repeated interactions over time, where people are given `\textit{active and supportive roles, [engaging] them in connecting to each others’ experiences, considering each other’s interpretations, and building insights that may lead to new meanings and relationships.}' The use of OurPlace within schools shows how this might be put into practice, with the local heritage around each school acting as each class's focus for both research and creativity. The nature of digital content and school cohort systems also encourages this to be an ongoing, living practice: where each class experiences and builds upon the previous class's local heritage research. As with Crivellaro's walking trail \citep{Crivellaro2016}, the mobile nature of the technology also supports a large amount of this process to happen through genuine engagement with the environment, with many of the app's Task Types encouraging learners to pause and reflect in-situ.

As McCarthy and Wright argue, mobile phones and tablets are intrinsically personal devices which are particularly well suited to engaging people in making intimate observations and reflections---allowing for private encounters in public space, and `\textit{blurring the traditional boundaries between public and private, intimate and extraneous}' \citep{McCarthy2005}. They argue that technologies which engage people on a personal level---rather than as an anonymous entity---can help them feel `in place'. This focus on the individual experience of using technology can allow for explorations of private interpretations of place within public space. For example, RIOT!1831 allowed participants to privately experience an interactive play, whilst in an authentic, yet public, space \citep{Blythe2006}. Similarly, Google Earth allowed for Gerstmann to explore personal experiences in a virtual, dream-like representation of real public spaces, with some of the more abstract elements being up to his interpretation \citep{Gerstmann2016}. Some of the usage of OurPlace mirrored these aspects of private experiences in and building of place: for example, one of the heritage workshop participants considered using OurPlace to subvert the usual bureaucratic system in place for choosing commemorative plaques, instead creating their own personal set of digital plaques independently. Other instances could be seen in schools' use of Activities, where students would retreat away from the main group to be able to record their thoughts and reflections without being interrupted or overheard. 

This focus on the individual is an important part of OurPlace---while many of the engagements involved participants creating Activities as a group, the application maintains the ability for individuals (be those members of the community, teachers or students) to use the technology as a platform for self expression within place, be that through creating their own Activities or responding to others'. Relph highlights the importance of individual stakeholders sharing their points of view: he argues that by moving away from recognising and representing individual viewpoints if favour of one which aims for wider representation, we run the risk of normalising `inauthentic' experiences of place by highlighting an identity which no longer authentically represents anyone \citep{Relph1976}. 

However, Relph also warns that place can be misrepresented, either by those who are invested in its success or blind to (or ignorant of) its flaws through what he dubs `museumisation': a process of simplifying and sanitising local history to create a more palatable, `disneyfied' ideal. Studies such as those held by Crivellaro et al. might suggest that this is best combated by opening up the ability for all individual stakeholders to create and share materials---during these studies, stakeholders took part in `anticipatory archaeology' to gave an authentic (if optimistic) representation of stakeholders' lived experiences \citep{Crivellaro2016}. However, this does not account for the fact that stakeholders cannot share what they do not know, and so may inadvertently create materials which do not give a complete representation of place. For example, Teacher 5's class created Activities relating to slavery abolitionists who had had a presence in the area---it is generally accepted that the North-East of England was a mainstay in the country for the abolitionist movement. However, the children were not aware that there had also been major businesses in the area that profited from the slave trade, including refineries of slave produced goods such as sugar and even ironworks which supplied slave restraints and plantation tools \citep{LitPhil2007}. In this way, the created Activities could be argued to give an incomplete (and rather charitable) representation of the place's heritage with regards to its relationship with slavery, and highlights a need for thorough research on the part of the content creator and a critical mind on the part of the consumer of these generated materials.

This project has highlighted that place-based mobile learning technologies can be useful tools for supporting place-making processes. While many digital technologies can open up new opportunities for encountering place, mobile technologies are unique in that they offer the ability to also do this in authentic physical \textit{and} social contexts, which has been argued to strengthen the learning and place-making experience. By supporting all users as creators of place-based mobile learning materials, these technologies can also be used to highlight the place attachment and place meaning held by stakeholders, both as individuals and in communities of practice. These materials can act as new layers of socio-technical infrastructures which grant visitors new opportunities for encountering place through novel interactions on a personal level. In this project, these technologies have been shown to have the potential to highlight place elements whose which stakeholders feel are underappreciated, to subvert the limitations of top-down institutions, and to allow for the sharing of lived experiences within place. However, representing place through mobile learning technologies can present the same potential pitfalls as other mediums, particularly when it comes to authentic and complete representation.

\section{Recommendations for Utilising Public Places as Infrastructure for Civic Mobile Learning}

\subsection{Genuine and Meaningful Engagement}

Relph asserts that by undervaluing places’ distinctive characteristics, being overly concerned with efficiency and accepting environments which are interchangeable to the point of anonymity, we run the risk of normalising‘inauthentic’experiences of place.

fox and le dantec - involve and emphasize the agency and perspectiveof community members from the outset, as they are the ones best positioned to inform thedesign process

OurPlace can be used for to create the same type of worksheet style tasks that blumenfeld disliked, as well as PBL. Community members had also recognised this (check that quote was included) 

\subsection{Support Stakeholder Requirements}

Previous work by Chatting et al. \citep{Chatting2017} has considered how an individual’s agency (argued to be a component of empowerment \citep{Ibrahim2007}) can be supported in the new Internet of Things world by applying lessons learned from the DIY movement to new, digital technologies. We believe that parallels can be drawn to this with our teachers and park volunteers: they were able to fulfil their goal through technology, (mostly) independent of the usual top-down institutional restrictions which would have affected their creative control and output (DG1). Uphoff argues that an empowerment process needs to provide access to ‘power resources’—the assets which create possibilities for achieving objectives [38]. For our teachers and volunteers, Activity authorship was a power resource: it allowed them to create their own content as they saw fit and release it in their own timeframe, with minimal top-down assistance. We suggest that future m-learning designs should consider how they can empower the user through content ownership. In ParkLearn, this was achieved by granting more creative control to users and elevating them from consumers to producers of educational content.

\subsection{Providing Sustainable Solutions}
working with and within stakeholder groups
balestrini note that a sense of ownership of the project wouldnot have been enough to sustain engagement with it—they argue that this was achieved byproviding value to all of the involved stakeholders (i.e. valuing the elderly participants’ lifeexperiences, giving the young students new technology skills, supporting the teachers inrunning an innovative educational project)
 facilitating a range of face-to-face social encounters can lead to discussion andongoing engagement
 Fox and Le Dantex - Community Historians - configure the project to be clearly and immediately advantageous to the community members, as collaborators rather than participants

\section{Limitations}

education
A limitation of this research is that they were only loosely tied into each school curriculum, and didn't assess learning outcomes. We hope that future engagements can offer deeper integration into existing curricula and feature formal assessments, supporting further insights into how PBML can fit into wider school systems. 

place stakeholders
To better understand these factors, it’s likely that we as researchers will have to growrelationships with a place over time through direct habitation, or at least involve existingstakeholders as participants who have ongoing relationships with it. However, this becomesboth more interesting and potentially problematic when accounting for Tuan’s arguments ofthe influence of personal experience–because a person’s place attachment is formed basedupon individual experience, researchers should be aware that their perceptions of place may not align with others’, running the risk of inappropriate design decisions if not handled with care

star vs workshop participants
Star recalls that few participants in one of her projects utilisedthe final system that her team designed, despite the researchers following the principlesof participatory design throughout the process.  They identified that this was not becauseof usability issues with the interface, but rather how their design was a poor fit with theinfrastructures the participants had to work with. 

digital civics technologies used to support automation/austerity measures

\section{Conclusion}

\subsection{Responding to Research Questions}

How can mobile learning technologies better surface and utilise the civic value of places and empower the communities which give them meaning?

\subsubsection{How can existing place and community infrastructures be better utilised as resources for mobile learning?}

\subsubsection{How can we design mobile technologies which promote civic learning?}

\subsubsection{How can we design mobile technologies which empower place stakeholders?}
