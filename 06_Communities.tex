\chapter{Communities' use of OurPlace}

\todo[inline]{Placeholder text from paper}

\section{Overview}

\section{Talking Statues}

We worked alongside two members (male and female, aged in their 60s) of the local park’s volunteer group, who had approached the researchers about wanting to produce a ‘talking statue’.  Built upon an existing monument of a key figure of the park’s history, this talking statue would share his story with visitors, encourage them to further explore the park and even to join the volunteer group. However, as the volunteers had very limited technical knowledge and funds, a bespoke digital technology seemed inaccessible to them. A physically wired system would have also been too expensive and would also have interfered with the monument’s status as a listed historical structure. 

Through ParkLearn, the volunteers created a talking statue Activity which
combined a ‘Listen to Audio’ Action (for which a volunteer wrote and read from a
script, g iving an account of the park’s history from the statue’s perspective)
with multiple ‘Location Hunt’ Actions, which gave the learner a playful guided
tour of the park (Table 2, row 10). The recorded audio was also transcribed
between multiple ‘Read Info’ Actions, which featured imagery from the park and
external links to the volunteer group’s website. Foamex signs featuring a QR
code (supplied by the ParkLearn website) were printed and attached to benches
near the statue (Figure 1). By making the Activity ‘private’ and using these
signs, the volunteers could ensure that only people near the statue could launch
the Activity. As this meant people would have to be present in the park to use
it, they treated the statue as an attraction, something that would raise the
profile of the park and encourage people to visit. They printed posters to
advertise the project to the surrounding community, and even talked to the local
press. 

After the launch of the installation, the volunteers were eager for regular
updates regarding its usage by park visitors. To facilitate, we updated the
ParkLearn website to show the number of times each Activity had been scanned (95
scans in the first 30 days). ParkLearn allowed the volunteers to create a
digital, multimedia instalment with minimal interaction or support from the
local council (from whom they required permission to put up the scan points).
Due to the use of pre-existing technology, the total cost of the installation
was around £50 (the cost of producing the Foamex signs). The talking statue
launched in the same summer in which it was conceived, rather than the original
target launch date of the following year. The actual creation of the Activity
took less than an hour, with assistance from a researcher (neither of the
volunteers had downloaded a mobile app before).

\section{An Ethnography within the Tyne \& Wear Heritage Forum}

\subsection{An Overview of the TWHF}

\subsection{Engagements and Workshops}

\section{Resulting Interest in and Uptake of OurPlace}

\section{Stakeholder Desires, Requirements and Tensions}

\section{Challenges Encountered}

\section{Summary}