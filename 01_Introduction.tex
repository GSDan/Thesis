\chapter{Introduction}

\epigraph{`An empathetic and compassionate understanding of the worlds beyond our own places may best be grounded in a love of a particular place to which I myself belong. In this way, we may recognize that what we need in our everyday world has parallels in the worlds of others.'}{\textit{Edward Relph}}

\section{Research Motivation and Context}
Smartphones have reached the point of ubiquity in the UK: with an estimated 82\% of the total population (around 55 million people) owning a smartphone in 2018, the country has one of the highest levels of smartphone penetration in the world \citep{wikipedia2020}. With this comes a similar level of ubiquity of access to information---thanks to mobile access to the Internet, people are now able to create and consume multimedia content on-demand, regardless of time or location.

This has presented a wealth of new opportunities for computer-assisted learning and further popularised the concept of `mobile learning', which Crompton et al. define as: `\textit{learning across multiple contexts, through social and content interactions using personal electronic devices}' \citep{Crompton2013}. As mobile devices have gained in ubiquity, functionality and computing power, the popularity and sophistication of mobile learning applications and websites has increased in turn. Hardware features such as GPS and camera systems are being used to deliver educational content such as augmented reality experiences \citep{google2020} and interactive quizzes that adapt to the learner's physical location \citep{Giemza2013}, enabling cross-media learning within authentic environments. The increasing availability of these rich learning experiences on tablet-sized devices has also led to the adoption of mobile learning within UK schools, with nearly half being expected to have one tablet per child within the next few years \citep{BritishEducationalSuppliersAssociation2015}. However, due to the need for advanced technical knowledge, the means to create bespoke versions of these rich mobile learning experiences has remained out of reach for many teachers and students.

This research also takes place against a backdrop of political and financial uncertainty for much of the UK's public sector. The combination of significant austerity measures on local government budgets and a renewed focus on localism has led many local authorities to increasingly rely upon volunteerism, in the stead of properly funding the maintenance of public spaces such as parks. As a result, community spaces are being increasingly cared for by groups of volunteer stakeholders. However, as volunteering is more attractive to the time-rich, these volunteer groups are largely made up of retirees, with young adults and those living in lower socio-economic groups being less likely to volunteer \citep{ncvo2019}. With the knowledge that these spaces rely on the ongoing support of volunteer stakeholders, volunteers are continuously looking at ways in which they can engage younger audiences in an effort to share their knowledge, highlight their perceived value of place, and increase their groups' long-term sustainability. 

With the increasing prevalence of mobile devices, these stakeholder groups are frequently turning to online digital presences, through mediums such as social media platforms and mobile applications (inspired by the success of platforms such as \textit{Pok\'emon Go}). However, the older demographics that constitute the majority of these groups typically have less experience and confidence with using digital technologies. This lack of digital literacy has meant these groups often struggle to effectively create and maintain group websites and social media pages, and, like teachers and students, lack the knowledge and resources necessary for creating bespoke mobile applications. As a result, these volunteer stakeholders frequently struggle to create engaging digital solutions through which to share their knowledge, passions and values with new audiences.

This thesis explores the design space for mobile platforms which support users in the creation of bespoke mobile learning activities which harness places and communities as learning resources---both in the formal education context of schools, and in the more informal context of community-generated mobile learning experiences. Of particular interest is how such technologies could be used by these community stakeholders to further their own agendas, and share their place-based knowledge and values. In an effort to understand how stakeholders' contexts, values and knowledge can form and be shared, this project frequently engages with the qualities of `place', and how one develops relationships with it. While ‘space’ and ‘place’ are often used interchangeably in everyday discourse, Yi-Fu Tuan argue that they have quite different meanings: while space might describe the physical properties of a location, place is a metaphysical concept created by human meaning attributed to that space \citep{Tuan1978}. He argues that \textit{place} goes beyond mere location---it also carries a social position. He posits: \textit{`What begins as undifferentiated space becomes place as we get to know it better and endow it with value'}. Thus, place is a social, spatio-temporal value, where space and time go together in shaping a person's interpretations. People who inhabit the same physical space may, due to differing past experiences, associate the space with different meanings and values. Spaces mean different things to different people---put simply, it's what makes a house a home. Relph posits that in order to be able to encourage the making of new places or the maintenance and restoration of new ones, we must first further understand how we experience both space and place and be able to describe what makes a place special \citep{Relph1976}. Dourish and Bell argue that our experiences with space are formed through encounters with social, institutional and historical layers of `infrastructure' \citep{Dourish2007}. They argue that these are the fundamental elements through which we encounter \textit{space} and form \textit{place}, and that highlighting these infrastructures serves as a method to understand the social and cultural practices that occur within a space. Through layers of infrastructure we experience the world and produce, understand and enact cultural meaning, and form relationships with space---a process frequently called `place-making'. While several previous HCI research projects have engaged with how digital technologies can act as mediators for place and promote place-making, this project will more specifically explore how user-generated content within mobile learning technologies can highlight and utilise these layers of infrastructure for knowledge and value sharing, as well as the empowerment of place stakeholders.  

In an effort to investigate how such tools and resources could be used within formal education, this thesis simultaneously explores how such technologies could provide value to schools, and---to assist in the utilisation of learning resources within the local community---how they could be used by teachers to support learning across multiple pedagogies and contexts. The benefits of learning outside of the traditional classroom context is widely documented, particularly in physical contexts relevant to the subject matter \citep{Fiennes2015, Ofsted2008}. Lave and Wenger go as far to argue that most people's learning does not actually occur in these traditional classroom settings, arguing that learning is normally situated: embedded within activities, contexts and cultures \citep{lave1991situated}. This goes beyond physical contexts and the qualities of space: they argue that a large amount of effective learning occurs through social interactions and exposure to communities of practice. While schools exist within wider communities boasting rich cultural contexts and both cultural and domain knowledge, it is rare that these potential resources are utilised within school curricula \citep{Leat2015}. 

This re-framing of citizens away from consumers and towards a more participatory model in which they are able to take an active role within their relationships with local authorities is a core tenet of the Digital Civics agenda \citep{Olivier2015}: a movement within which this project is situated. Several existing Digital Civics projects have explored how digital technologies could help schools utilise domain expertise within local communities \citep{Dodds2017} and institutions \citep{Megan2019}, as well as promoting cultural exchange through school communities \citep{Sarangapani2016}. Mobile learning technologies have been shown to be effective at engaging with physical learning contexts such as museums \citep{Lonsdale2004}, and even enabling the seamless transition of learning between multiple physical contexts \citep{Wong2011}. Historically, however, mobile learning technologies have struggled to engage with social learning contexts  \citep{Frohberg2009}, meaning that the infrastructures of place have gone under-explored as resources for mobile learning. Therefore, this thesis also explores how mobile learning technologies can be used as tools with which schools can utilise places and communities as learning resources, through various pedagogical approaches.


\section{Research Question and Objectives}
\label{sec:ResearchQuestions}

This research is exploratory in nature, following a design-based research approach (detailed in Section \ref{sec:ResearchApproach}). As a result, the specific aims of each engagement were often dependent upon the findings of the previous studies up until that point. That said, there exists a larger, over-arching research question to guide the project as a whole, formed in response to the findings resulting from the engagements detailed in Section \ref{sec:Parks2026}. This main research question is:

\begin{displayquote}
\textit{\textbf{In what ways can mobile learning technologies better surface and utilise the civic value of places and empower the communities which give them meaning?}}
\end{displayquote}

As the scope of this question makes it somewhat unmanageable when approached as a whole, I have split it into three core research objectives:

\begin{enumerate}

\item \textit{\textbf{Investigate how existing place and community infrastructures can be better utilised as resources for mobile learning.}}

This thesis aims to fulfill this objective by:

\begin{itemize}
  \item Identifying social and physical place infrastructures which can be (or are) utilised as learning resources.
  \item Investigating what factors may be limiting the use of these learning resources within educational contexts.
  \item Exploring ways in which mobile learning technologies could mitigate these limiting factors.
  \item Designing, deploying and evaluating mobile learning technologies which utilise these place infrastructures as learning resources within formal and informal educational contexts.
\end{itemize}

\item \textit{\textbf{Explore how mobile learning technologies can be designed to promote civic learning.}}

This thesis aims to fulfill this objective by:

\begin{itemize}
  \item Exploring how place infrastructure can be utilised as civic learning resources.
  \item Defining a design space for civic mobile learning technologies using these infrastructures.
  \item Investigating how this design space can be addressed through iterative technology deployments.
\end{itemize}

\item \textit{\textbf{Explore how mobile learning technologies can be designed for the empowerment of place stakeholders.}}

This thesis aims to fulfill this objective by:

\begin{itemize}
  \item Examining how prior HCI research has designed technologies for stakeholder empowerment.
  \item Identifying ways in which mobile learning technologies could act as a vector for stakeholder empowerment.
  \item Deploy mobile learning technologies within community groups, and examine their usage by stakeholders in fulfilling their own needs and agendas.
\end{itemize}

\end{enumerate}

The studies detailed in this thesis aim to meet these objectives. The results of these engagements are summarised in relation to the above research objectives in Section \ref{sec:RespondingtoQuestions}.

\section{Research Approach and Methods}
\label{sec:ResearchApproach}

This work is of an exploratory nature, following a largely qualitative approach. This has been chosen due to the importance of context in the project: examining, understanding and designing for stakeholders' lived experiences. Qualitative methodologies aim to describe and understand lived experience, rather than to predict and control an objective and generalizable reality---which, when dealing with personal experiences, may not actually exist \citep{macdonald2012}. Many of the topics broached over the course of the project are extremely context-dependent: for example, in comparison to more immutable school subjects, such as maths and physics, students across different contexts are likely to have different learning experiences about community heritage and citizenship. The same is likely true for community stakeholders working in different areas of interest and socioeconomic contexts. As such, I needed to appropriate a research approach which supported methods such as design workshops and semi/unstructured interviews for gaining a greater understanding of the design contexts and stakeholder requirements, as well as practical interventions and technology deployments in authentic contexts to develop theory and design recommendations. As Brown argues, if one believes that context matters, disregarding it to assess learning in laboratory settings will only result in an incomplete understanding of these processes in more naturalistic contexts \citep{brown1992}. As such, I wanted the research undertaken during this project---both with schools, and with groups of community stakeholders---to be strongly positioned within the participants' current lived contexts.

Until recent years, situating educational research within real-world (i.e. school classroom) contexts was somewhat uncommon. In the 1990s, there was an increasing frustration with lab-based educational research, and how disconnected it had become from teaching practice. There was an acknowledgement that conflicting stakeholder perspectives could not be modelled objectively through reductionist, lab-based processes. Reeves argues that for academic educational research to be of practical use, `\textit{practitioners must be more directly engaged in the conduct of [it]}' \citep{reeves2000}. In the years following this, numerous research approaches have gained in popularity which aim to develop and assess designs in context, often alongside practitioners and other stakeholders. 

One of these research approaches which could potentially be used for this project is `action research' (AR): an approach which seeks to engage with the complex dynamics involved in a given social context, rather than look more more generalizable explanations. The purpose of AR is to impart social change, through the development and assessment of specific actions and professional/community practices \citep{macdonald2012}. Furthermore, researchers are encouraged to recognise that engaging in these practices often necessitates also engaging with complex socio-economic contexts \citep{stringer2013}. As such, action research is grounded in authentic contexts, where the researcher also often acts as the domain expert (e.g. teacher, community stakeholder) during interventions. As Hayes notes, in order to allow for the consideration of these contexts, AR methodology is often open-ended and iterative---the primary focus is to implement action (e.g. policy or process changes, the introduction of technology), and the work is judged by the quality of the research results and the feasibility of any solutions that emerge \citep{hayes2011}. However, critics of action research argue that this focus upon implementing change is a weakness of action research: Reeves argues that `\textit{[action research features] little to no effort to construct theory, models, or principles to guide future design initiatives}', instead being limited to solving specific problems in specific places \citep{reeves2000}. Action research goals are usually focused on a particular program, product, or method, in order to describe, assess and incrementally improve it in context. As such, Reeves notes that such approaches can be seen as a form of evaluation for interventions designed for specific contexts, rather than a research process which produces more broadly applicable design theory. As I wanted to produce contributions which could be still be easily applicable outside of specific contexts and the application of particular technologies, I decided against an action research approach.

In comparison, `design-based research' (DBR, also known as `design research', `design experiments') places a much greater focus on the practical generation and application of theory rather than context-specific artefacts, while still remaining framed in the real world \citep{zimmerman2007}. The research approach can be summarised by three main characteristics: through collaboration with practitioners, it addresses complex problems within real-world contexts (i.e. classrooms); it produces plausible solutions to these complex problems by integrating existing and hypothesised design principles with the use of technologies; and produces and tests these hypothetical design principles through conducting, reflecting upon and refining innovative learning environments \citep{reeves2000, brown1992, collins1992}. In short, the goal of this approach is to solve real-life problems, while simultaneously reflecting upon the results to construct design principles for potential use in other contexts. As such, DBR supports a model of research approach which intends to produce new learning theories, artefacts (e.g. technologies) and practices within naturalistic settings---focusing on understanding the messiness of real-world practice by treating context as a core research focus \citep{Barab2004}. Cobb et al. argue that DBR involves `engineering' (through the active participation of the researcher) forms of learning through interventionist methods which involve some sort of design, and then studying those forms of learning within naturalistic contexts, iteratively revising the context and design in response to results \citep{cobb2003}. Furthermore, Barab and Squire argue that during DBR practices, participants are treated closer to co-researchers than subjects---helping produce and develop new practices, rather than the practices simply being done to them \citep{Barab2004}. Solidifying this and demonstrating that DBR can be an effective approach within the scope of mobile learning research, Herrington et al. describe how they used DBR to develop design principles for the use of mobile learning technologies within schools: they note that the early stages involved in-depth consultations with practitioners, before then moving onto the iterative development of interventions and theory \citep{herrington2009}.

However, DBR projects need to account for and adapt to the messiness and complexity of the given research context(s), while still producing findings which will be valuable to others outside of that context. There needs to be a balance between designing for the immediacies of the specific and the overly abstracted, with the findings extending beyond the particulars of a given context. As Barab and Squire argue, design-based research `\textit{strives to generate and advance a particular set of theoretical constructs that transcends the environmental particulars of the contexts in which they were generated, selected, or refined}.' They also argue that the theories resulting from DBR should be justified and evidenced through impact generated by the research within the local context, noting: `\textit{Design-based research that advances theory but does not demonstrate the value of the design in creating an impact on learning in the local context of study has not adequately justified the value of the theory.}' Another potential issue is that the added focus upon the impact of local contexts means that these contexts, and the findings of the studies which take place within them, can rarely be reproduced with any accuracy. As a result, replicability is far more difficult in design-based research than in lab-based research methods. In order to mitigate this, projects need to `problematize' the process as transparently as possible, providing rich and detailed descriptions of the artefacts, interventions, participants and contexts to support peers in understanding \textit{why} results occurred. In this way, DBR should iteratively develop and test theory in authentic contexts---validating not just the tested design, but the theoretical constructs upon which the design's goals were based. Barab and Squire argue that these theories should be adaptable so that they can be applied in other contexts, reducing the need to "sterilize" context for the sake of replicability.

Due to the nature of the research subject matter being so intrinsically linked to local context, this project attempts to follow a design-based research methodology. As such, these working contexts are included as a core feature of the studies' engagements and findings, as they are likely to affect how participants approach and use mobile technologies and any frameworks produced to support them. In order to `problematize' these contexts through rich and detailed descriptions, this project favours in-depth and long-term engagements with local stakeholders and schools where possible. These engagements are impact focused, rather than driven by the collection of quantitative data: assessing the performance of theory-driven designs for iteration and further validation in multiple contexts.

As noted by Herrington et al., design-based research projects take place over several stages: problem analysis and stakeholder \textit{consultation}; the \textit{design} of an initial intervention; the implementation of this \textit{intervention} over multiple iterations, with \textit{adjustments} and improvements made between deployments; and the creation of design principles based upon the theory, practice and \textit{reflection} of the previous phases \citep{herrington2009}. 

Due to the variety of different engagement contexts and engagement lengths, a variety of exploratory, qualitative research methods will be used during the \textit{consultation} stage. The methods used in this stage include semi and unstructured interviews, focus groups, participant workshops, and deployments with technology probes. The method of coming into contact with each participant group will be noted when appropriate, as each differ according to context. When other more specific and/or one-off methods are used during an engagement (e.g. particular workshop activities), they will be discussed in this document as and when appropriate. 

The \textit{design} stage will consist upon taking the combined findings from a literature review and the consultation stage to produce a series of design goals for an appropriate mobile learning technology. An initial version of this technology will then be produced with these goals in mind. 

The \textit{intervention} stage will consist of the repeated deployment and assessment of the technology's design, with a variety of stakeholders in numerous contexts. These stakeholders will include school students, teachers and volunteers within community groups. Unless otherwise noted, a member of the research team (usually just myself) will be present for all engagements and technology deployments. Observations will be recorded through field notes whenever possible, however these engagements will frequently require me to lead sessions (e.g. running workshops, instructing classroom activities), provide technical support (i.e. in the event of software malfunctions or users requiring support), or simply feature too many participants for me to keep track of. In these cases, additional details and impressions will be obtained from co-organisers, such as teachers and community stakeholders. To gain a more detailed understanding of the intervention's performance and the impact of the deployment context, additional impressions will also be gathered from participants after each intervention. This will be preferably be done through semi-structured interviews of around 30 minutes in length, held either immediately following each study or, if that isn't practical, in the days following. These interviews will be audio recorded, and typically aim to gain an understanding of the participant's opinions on how that particular engagement went, what could be improved about the technology or process, and the reasoning behind any decisions that the participant might have made.

Between these interventions, the technology's design will be \textit{adjusted} according to each deployment's findings and stakeholder feedback. Finally, implications for design will be produced as a result of \textit{reflection} upon the data and findings from the previous stages. Audio recordings will be listened to before being transcribed---either fully, by myself or through external professional and trusted transcription services, or only partially by myself if the contained lots of `dead air' or non-pertinent conversations (e.g. wind noise during school trips, participant chatter about personal issues). These transcriptions will be collated along with other sources of data (e.g. notes, photos) per engagement and analysed through a inductive thematic analysis, which consisting of exploratory, line-by-line coding. These codes will then then grouped into themes pertinent for discussion in this document.

All interactions and data collection have been approved by and conform to the requirements of Newcastle University’s ethics committee. Consent forms are required to be completed by each participant (or, in the case of schools, students' parents) prior to the commencement of each study. Accompanied by an information sheet (see Appendix \ref{app:infoSheets}) detailing the purpose and details of each study as well as contact details of the research team, these consent forms (see Appendix \ref{app:consentForms}) ask for affirmative consent for a number of factors, including: general participation in the study, an understanding that they can withdraw consent at any time, consent for capturing photographs of the participant, and consent for capturing audio recordings. Participants are told that their consent for each of these could be granted, denied or withdrawn at any time. Any identifiable elements (i.e. names, photographs) have been anonymised prior to publishing. In accordance with Newcastle University's ethics policies, participant data is securely stored on Microsoft OneDrive.

\section{Summary of Contributions}
The course of conducting the investigations discussed in this thesis has resulted in a number of contributions being made to the field of Human Computer Interaction. In order of presentation in this thesis document, these contributions are:

\begin{itemize}
    \item Insights from several years' worth of a variety of engagements with multiple place stakeholders, including: embedded, multi-year relationships with volunteer-led organisations; longitudinal and short studies with teachers and students from seven different schools; and one-off technology deployments and public workshops, ranging from three to fifty participants in size.
    \item The introduction of a model for a social design space for mobile learning technologies, where relationship infrastructures connect stakeholders and learners in space and place. The model illustrates how `traditional' mobile learning approaches don’t meaningfully engage with these infrastructures, and are either independent of the learner’s context or concentrate solely on the physical aspects of the environment. Implications for designing technologies which aim to support place-making within this space are also provided.
    \item The design and development of OurPlace: an open-source mobile learning platform designed to support teachers, students and community stakeholders in creating, sharing and engaging with bespoke mobile learning activities seamlessly, across multiple learning contexts. This document shows OurPlace to be an adaptable research tool, and it has been used in three (as of the time of writing) other projects held by different researchers, each of which engage either with schoolchildren or adult community stakeholders.
    \item Reports on the use of OurPlace in both formal and informal learning contexts, with discussions around how it and similar technologies can promote civic engagement and inquiry; support empowerment through encouraging creativity and content ownership; and assist in seamless learning teaching practices by being an adaptable, supporting toolkit.
    \item An introduction and exploration of the concept of `project-based mobile learning' (PBML) through the creation, application, and iteration of a PBML framework in four different schools and a summer school of Travelling Showmen. Suggestions for the framework's reconfiguration in response to contextual challenges are also contributed, along with reflections on the PBML process, including how PBML harnessed students’ existing desires for independence, and how it could offer new avenues for leveraging place as a learning resource.
    \item Implications for design regarding how mobile learning technologies can be configured to support place-making, and recommendations as to how researchers and designers can better utilise the infrastructures of place as resources for civic mobile learning.
    \item This project has also had a notable impact within the North East region of England:
    \begin{itemize}
    \item Approximately 400 schoolchildren have engaged with the technology, providing opportunities for learning outside of the classroom---frequently in authentic place and/or engaging with community stakeholders.
    \item Multiple stakeholder groups have used the technology to create mobile learning activities, trails and installations.
    \item A Community Rail Partnership has successfully applied for grants to fund the creation of a job dedicated to using OurPlace, running engagements with children and local communities centred along a railway line in the North of England.
    \end{itemize}
\end{itemize}

By their nature, many of the insights and contributions from these works are heavily contextual. However, as discussed earlier, the employed design-based research and reporting approach means that the majority of the findings and discussions should be adaptable for use in other contexts. Furthermore, the PBML framework and the presented implications for design have been deliberately configured to support their application in different contexts and with different technologies.

\section{Document Structure}

The remainder of this thesis consists of eight chapters, each concerning a different subject or series of studies: 

\textbf{Chapter \ref{chap:SpacePlaceInfrastructure}} gives a brief introduction to prior literature produced by humanist geographers and HCI researchers relating to our relationships with place, and how technology can influence these relationships. Subjects covered include: different interpretations of place, and how it differs from space; ways in which individuals' relationships with place can be experienced and described; the concept of place-making, and how relationships with place can be developed; how technology can and has been used to promote the development of these relationships in HCI research; and an overview of Digital Civics projects which have engaged with place-making.

\textbf{Chapter \ref{chap:MobileLearning}} gives a brief overview of some of the previous research that has been conducted concerning mobile learning, constructionism, and project-based learning. Subjects covered include: situated learning and communities of practice; civic learning; an overview of some existing mobile learning research; ways in which the mobile learning technologies have engaged with the infrastructures which make up place; mobile technologies supporting learning seamlessly across multiple contexts; and the use of mobile learning technologies to support constructionist and project-based learning pedagogies.

\textbf{Chapter \ref{chap:DesignSpace}} covers the first series of engagements, which aimed to gain an understanding of the potential for mobile technologies to explore the different stakeholders’ current issues and practices; explore how these can be used as resources for civic learning; and develop generalizable design requirements for future technologies for m-learning within civic space. The findings of these engagements culminate in a model of the design space for civic mobile learning technologies, along for implications for designing technologies within this space.

\textbf{Chapter \ref{chap:Design}} presents a number of design goals for a place-based, mobile learning technology for use within formal and informal learning contexts. These goals were formulated in response to the studies and literature covered up until this point. The chapter then gives a detailed overview of OurPlace---the mobile learning platform created in response to these design goals.

\textbf{Chapter \ref{chap:Community}} describes an multiyear ethnographic study with a local heritage forum, and the engagements which came as a result of it. The findings of these engagements are discussed, particularly around the role of technology in satisfying stakeholders' desires, requirements and the potential tensions that can exist between technology, stakeholders, and the places they care about. These findings were also used to assess if OurPlace met the original design goals relating to local stakeholders that were set out in Chapter \ref{chap:Design}.

\textbf{Chapter \ref{chap:Teachers}} covers studies in formal education settings, most of which ran concurrently to those covered in Chapter \ref{chap:Community}. In these studies, teachers and researchers created OurPlace content to be completed by school students across multiple engagements, investigating the use of OurPlace as a seamless, place-based learning tool. The findings of these studies are presented and discussed through themes relating to seamless learning practices, engaging and empowering learners through control and content ownership, and the use of mobile learning technologies for civic engagement and inquiry. These findings were also used to assess if OurPlace met the design goals relating to learning in formal education contexts.

\textbf{Chapter \ref{chap:student-created}} explores the use of mobile learning platforms such as OurPlace in supporting project-based learning pedagogies, where students take the role of Activity designer and the application is used as a component within a larger project. The concept of `project-based mobile learning' (PBML) is introduced through the creation and application of a PBML framework. This chapter covers the framework, suggestions for its configuration in response to contextual challenges, reflections on how PBML can harness students’ existing desires for independence, and how it could offer new avenues for leveraging place as a learning resource. 

\textbf{Chapter \ref{chap:discussionConclusion}} discusses the findings identified during the previously discussed studies, with discussions pertaining to how mobile learning technologies can be configured to support place-making, and recommendations as to how researchers and designers can better utilise the infrastructures of place as resources for civic mobile learning. I also discuss some of the study’s limitations and respond to the research questions presented in Section \ref{sec:ResearchQuestions}.