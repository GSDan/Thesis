\chapter{Introduction}

\epigraph{`An empathetic and compassionate understanding of the worlds beyond our own places may best be grounded in a love of a particular place to which I myself belong. In this way, we may recognize that what we need in our everyday world has parallels in the worlds of others.'}{\textit{Edward Relph}}

\section{Project Context and Motivation}

Smartphones have reached the point of ubiquity in the UK: with an estimated 82\% of the total population (around 55 million people) owning a smartphone in 2018, the country has one of the highest levels of smartphone penetration in the world \citep{wikipedia2020}. With this comes a similar level of ubiquity of access to information---thanks to mobile access to the Internet, people are now able to create and consume multimedia content on-demand, regardless of time or location.

This has presented a wealth of new opportunities for computer-assisted learning and further popularised the concept of `mobile learning', which Crompton et al. define as: `\textit{learning across multiple contexts, through social and content interactions using personal electronic devices}' \citep{Crompton2013}. As mobile devices have gained in ubiquity, functionality and computing power, the popularity and sophistication of mobile learning applications and websites has increased in turn. Hardware features such as GPS and camera systems are being used to deliver educational content such as augmented reality experiences \citep{google2020} and interactive quizzes that adapt to the learner's physical location \citep{Giemza2013}, enabling cross-media learning within authentic environments. The increasing availability of these rich learning experiences on tablet-sized devices has also led to the adoption of mobile learning within UK schools, with nearly half being expected to have one tablet per child within the next few years \citep{BritishEducationalSuppliersAssociation2015}. However, due to the need for advanced technical knowledge, the means to create bespoke versions of these rich mobile learning experiences has remained out of reach for many teachers and students.

This research also takes place against a backdrop of political and financial uncertainty for much of the UK. Largely in response to 2008's global financial crisis, the United Kingdom's Conservative government implemented an extended period of austerity. The impacts of these policies have been wide-reaching and long-lasting, affecting everything from healthcare to education. With the combination of significant cuts to local government budgets and a renewed focus on localism, many local authorities have increasingly relied upon volunteerism in the stead of properly funding the maintenance of public spaces such as parks.

As a result, community spaces are being increasingly cared for by groups of volunteer stakeholders. However, as volunteering is more attractive by nature to the time-rich, these groups are largely made up of retirees, with young adults and those living in lower socio-economic groups being less likely to volunteer \citep{ncvo2019}. With the knowledge that the spaces that they care about rely on the ongoing support of volunteer stakeholders, these groups are continuously looking at ways in which they can engage younger audiences in an effort to share their knowledge, highlight their perceived value of place, and increase their groups' long-term sustainability. 

With the increasing prevalence of mobile devices, these groups are frequently turning to online digital presences, through mediums such as social media platforms and mobile applications (inspired by the likes of \textit{Pok\'emon Go}). However, individuals within the older demographics that constitute the majority of these groups typically have less experience and confidence with using digital technologies (often labelled as `digital literacy'). Because of this lack of technical knowledge, these groups often struggle to effectively create and maintain group websites and social media pages, and, like teachers and students, lack the knowledge and resources necessary for creating bespoke mobile applications. As a result, these stakeholder groups frequently struggle to create engaging digital solutions through which to share their knowledge, passions and values with new audiences.

This thesis explores the design space for the development of mobile platforms designed to support users in the creation of bespoke mobile learning activities which harness places and communities as learning resources---both in the formal education context of schools, and in the more informal context of community-generated mobile learning experiences. Of particular interest is how such technologies could be used by these community stakeholders to further their own agendas, and share their knowledge and civic values. Furthermore, this thesis also explores how these technologies could be used by teachers to support seamless learning across multiple pedagogies and contexts (supporting both learning in authentic place and more traditional classroom settings).

\section{Research Questions}
\label{sec:ResearchQuestions}

The research in this thesis is exploratory in nature, following a design-based research approach (detailed below, in Section \ref{sec:ResearchApproach}). As a result, the specific aims of each engagement were largely reactionary and dependent upon the findings of the previous studies up until that point. That said, there exists a larger, over-arching research question to guide the project as a whole, formed in response to the findings resulting from the engagements detailed in Section \ref{sec:Parks2026}. This main research question is:

\begin{displayquote}
\textit{\textbf{How can mobile learning technologies better surface and utilise the civic value of places and empower the communities which give them meaning?}}
\end{displayquote}

As the scope of this question makes it somewhat unmanageable, I have split it into three more specific questions:

\begin{displayquote}
\textit{\textbf{How can existing place and community infrastructures be better utilised as resources for mobile learning?}}
\end{displayquote}

\begin{displayquote}
\textit{\textbf{How can we design mobile technologies which promote civic learning?}}
\end{displayquote}

\begin{displayquote}
\textit{\textbf{How can we design mobile technologies which empower place stakeholders?}}
\end{displayquote}

The studies detailed in this thesis explore the issues surrounding these questions, and the results of these engagements are used to respond to each question in Section \ref{sec:RespondingtoQuestions}.

\section{Research Approach}
\label{sec:ResearchApproach}

Many of the topics broached over the course of the project, such as individuals' relationships with place or the roles of citizens within communities, are extremely context-dependent: for example, in comparison to more immutable school subjects, such as maths and physics, students across different contexts are likely to have different learning experiences about community heritage and citizenship. The same is likely true for community stakeholders working in different areas of interests and socio-economic contexts. For these reasons, it would be unsuitable to assess learning outcomes as isolated variables in a traditional, quantitative manner: as Brown argues, if one believes that context matters, disregarding it to assess learning in laboratory settings will only result in an incomplete understanding of these processes in more naturalistic contexts \citep{brown1992}. As such, the research undertaken during this project---both with schools, and with groups of community stakeholders---is strongly positioned within the participants' current lived contexts (be that physical, political, socio-economic, etc).

However, as the project has a pro-active research agenda involving the design and deployment of new mobile learning technologies and ways of using them, simply observing current practices within context, while important and valuable, is not enough. Instead, as a researcher, I must become involved: participate in the practices of the teachers and stakeholders in order to deploy, observe the performance of and then refine designs and processes within real-world contexts. This requirement is in-line with design-based research (DBR), a model of research approach which intends to produce new learning theories, artefacts (e.g. technologies) and practices within naturalistic settings---focusing on understanding the messiness of real-world practice by treating context as a core research focus \citep{Barab2004}. Cobb et al. argue that design-based research involves `engineering' (through the active participation of the researcher) forms of learning through interventionist methods which involve some sort of design, and then studying those forms of learning within naturalistic contexts, iteratively revising the context and design in response to results \citep{cobb2003}. Furthermore, Barab and Squire argue that during DBR practices, participants are treated closer to co-researchers than subjects---helping produce and develop new practices, rather than the practices simply being done to them \citep{Barab2004}.

However, DBR approaches do still present some difficulties. For example, DBR projects need to account for and adapt to the messiness and complexity of the given research context(s), but still produce findings which will be valuable to others outside of that context. There needs to be a balance between designing for the immediacies of the specific and the overly abstracted, with the findings extending beyond the particulars of a given context. As Barab and Squire argue, design-based research `\textit{strives to generate and advance a particular set of theoretical constructs that transcends the environmental particulars of the contexts in which they were generated, selected, or refined}.' They also argue that the theories resulting from DBR should be justified and evidenced through impact generated by the research within the local context, noting: `\textit{Design-based research that advances theory but does not demonstrate the value of the design in creating an impact on learning in the local context of study has not adequately justified the value of the theory.}' Another potential issue is that the added focus upon the impact of local contexts means that these contexts, and the findings of the studies which take place within them, can rarely be reproduced with any accuracy. As a result, replicability---a core tenant of scientific study---is far more difficult in design-based research than in lab-based research methods. In order to mitigate this, projects following DBR methods need to `problematize' the design, processes and contextual dynamics as transparently as possible, providing rich and detailed descriptions of the artefacts, interventions, participants and contexts to support peers in understanding \textit{why} results occurred. In this way, DBR should iteratively develop and test theory in authentic contexts---validating not just the tested design, but the theoretical constructs upon which the design's goals were based. Barab and Squire argue that these theories should be adaptable so that they can be applied in other contexts, reducing the need to "sterilize" context for the sake of replicability.

Due to the nature of the research subject matter being so intrinsically linked to local context, this project attempts to follow a design-based research methodology. As such, these working contexts are included as a core feature of the studies' engagements and findings, as they are likely to affect how participants approach and use mobile technologies and any frameworks produced to support them. To gain a deeper understanding of these contexts---in order to `problematize' them through rich and detailed descriptions, as suggested by Barab and Squire---this project favours in-depth and long-term engagements with local stakeholders and schools where possible. However, this was not always feasible: especially in the case of schools, which face restrictions around assessments, required curricula and term times. In the cases where only one-off or short-term engagements are possible, these contexts are still detailed and accounted for, in order to allow for a greater understanding of the technology's comparative performance within the contexts of longer-term engagements. These engagements are impact focused, rather than driven by the collection of quantitative data, assessing the performance of theory-driven designs for iteration and further testing and validation in multiple contexts.

\subsection{Data Collection and Analysis}

Due to the variety of different engagement contexts, engagement lengths, and the reactionary nature of iterating upon technologies and frameworks in response to findings, a variety of different research methods are used throughout the project. When a particular method is used during an engagement (e.g. particular workshop activities) it is discussed as and when appropriate. However, this section will give a brief overview of a number of research methods which were used fairly consistently for data collection and analysis throughout the project.

All interactions and data collection were approved by and conformed to the requirements of Newcastle University’s ethics committee. Consent forms were required to be completed by each participant (or, in the case of schools, students' parents) prior to the commencement of each study. Accompanied by an information sheet detailing the purpose and details of each study as well as contact details of the research team, these consent forms asked for affirmative consent for a number of factors, including: overall consent for general participation in the study, an understanding that they can withdraw consent at any time, consent for capturing photographs of the participant, and consent for capturing audio recordings of the participant. Participants were told that their consent for each of these could be granted, denied or withdrawn at any time.

Unless otherwise noted, a member of the research team (usually just myself) was present for all engagements and technology deployments. I took field notes whenever possible, however these engagements frequently required me to lead sessions (e.g. running workshops, instructing classroom activities), provide technical support (i.e. in the event of software malfunctions or users requiring support), or simply featured too many participants for me to keep track of. In these cases, additional details and impressions were obtained from co-organisers, such as teachers and community stakeholders. This was most often done through semi-structured interviews, which were typically around 30 minutes in length, and were held either immediately following each study or, if that wasn't practical, in the days following. These interviews were audio recorded, and typically aimed to gain an understanding of the participant's opinions on how that particular engagement went, what could be improved about the technology or process, and the reasoning behind any decisions that the participant might have made.

In accordance with Newcastle University's ethics policies, participant data (e.g. photos and audio taken by the research team) was securely stored---in this case, on Microsoft OneDrive. These audio recordings were then listened to before being transcribed---either fully, by myself or through external professional and trusted transcription services, or only partially by myself if the contained lots of `dead air' or non-pertinent conversations (e.g. wind noise during school trips, participant chatter about personal issues during a workshop wrap-up). These transcriptions were then collated along with other sources of data (e.g. notes, photos) per engagement and analysed through a inductive thematic analysis which consisted of exploratory, line-by-line coding. These codes were then grouped into themes pertinent for discussion in this document. In-keeping with DBR methodology, this analysis was fed back into the project to inform the iterative development of the project's technologies and processes.

\section{Summary of Contributions}
The course of conducting the investigations discussed in this thesis has resulted in a number of contributions being made to the field of Human Computer Interaction. In order of presentation in this thesis document, these contributions are:

\begin{itemize}
    \item Insights from several years' worth of a variety of engagements with a multiple place stakeholders, including: embedded, multi-year relationships with volunteer-led organisations, longitudinal studies with teachers and students from multiple schools, one-off technology deployments and workshops ranging from three to fifty participants in size.
    \item The modelling of a design space for civic mobile learning technologies, along with implications for designing such technologies which aim to support place-making.
    \item OurPlace: a seamless mobile learning platform designed to support teachers, students and community stakeholders in creating, sharing and engaging with bespoke mobile learning activities.
    \item Reports on the use of OurPlace in both formal and informal learning contexts, with discussions around how it and similar technologies can promote civic engagement and inquiry; support empowerment through encouraging creativity and content ownership; and assist in seamless learning teaching practices by being an adaptable, supporting toolkit.
    \item OurPlace has been shown to be an adaptable research tool, and (as of the time of writing) has been used in three other projects held by different researchers, each of which engage either with schoolchildren or adult community stakeholders.
    \item An introduction and exploration of the concept of `project-based mobile learning' (PBML) through the creation, application, and iteration of a PBML framework in four different schools and a summer school of Travelling Showmen. Suggestions for the framework's reconfiguration in response to contextual challenges are also contributed, along with reflections on the PBML process, including how PBML harnessed students’ existing desires for independence, and how it could offer new avenues for leveraging place as a learning resource.
    \item This project has also had a notable impact within the North East region of England:
    \begin{itemize}
    \item Approximately 400 schoolchildren have engaged with the technology, providing opportunities for learning outside of the classroom---frequently in authentic place and/or engaging with community stakeholders.
    \item Multiple stakeholder groups have used the technology to create mobile learning activities, trails and installations.
    \item A Community Rail Partnership has successfully applied for grants to fund the creation of a job dedicated to using OurPlace, running engagements with children and local communities centred along a railway line in the North of England.
    \end{itemize}
\end{itemize}

\section{Document Structure}

The remainder of this thesis consists of eight chapters, each concerning a different subject or series of studies: 

\textbf{Chapter \ref{chap:SpacePlaceInfrastructure}} gives a brief introduction to prior literature produced by humanist geographers and HCI researchers relating to our relationships with place, and how technology can influence these relationships. Subjects covered include: different interpretations of place, and how it differs from space; ways in which individuals' relationships with place can be experienced and described; the infrastructures which make up place, and how these have been studied by HCI researchers; the concept of place-making, and how relationships with place can be developed; how technology can and has been used to promote the development of these relationships in HCI research; and an introduction to the Digital Civics agenda, with an overview of Digital Civics projects which have engaged with place-making and/or the infrastructures of place.

\textbf{Chapter \ref{chap:MobileLearning}} moves into the field of education, and gives a brief overview of some of the previous research that has been conducted concerning mobile learning, constructionism, and project-based learning. Subjects covered include: situated learning and communities of practice; civic learning; Activity Theory, and the Task Model for Mobile Learning; an overview of some existing mobile learning research; ways in which the mobile learning technologies have engaged with the infrastructures which make up place; mobile technologies supporting learning seamlessly across multiple contexts; and the use of mobile learning technologies to support constructionist and project-based learning pedagogies.

\textbf{Chapter \ref{chap:DesignSpace}} covers the first series of engagements, which were used to identify the research domain and investigate the potential for technologies to support civic learning activities in public spaces. The chapter discusses eight months of engagements, which aimed to gain an understanding of the potential for mobile technologies to explore the different stakeholders’ current issues and practices; explore how these can be used as resources for civic learning; and develop generalizable design requirements for future technologies for m-learning within civic space. The findings of these engagements culminate in a model of the design space for civic mobile learning technologies, along for implications for designing technologies within this space.

\textbf{Chapter \ref{chap:Design}} presents a number of design goals for a place-based, mobile learning technology for use within formal and informal learning contexts. These goals were formulated in response to the studies and literature covered up until this point. The chapter then gives a detailed overview of OurPlace---the mobile learning platform created in response to these design goals.

\textbf{Chapter \ref{chap:Community}} investigates the use of the OurPlace platform by place stakeholders, as a medium through which to share their knowledge and values and further their own agendas. This chapter describes an multiyear ethnographic study with a local heritage forum, and the engagements which came as a result of it. These engagements include several deployments of OurPlace with various stakeholder groups, as well as multiple public workshops. The findings of these engagements are discussed, particularly around the role of technology in satisfying stakeholders' desires, requirements and the potential tensions that can exist between technology, stakeholders, and the places they care about. These findings were also used to assess if OurPlace met the original design goals relating to local stakeholders that were set out in Chapter \ref{chap:Design}.

\textbf{Chapter \ref{chap:Teachers}} covers studies in formal education settings, most of which ran concurrently to those covered in Chapter \ref{chap:Community}. In these studies, teachers and researchers created OurPlace content to be completed by school students across multiple engagements, including a longitudinal study with a local primary school and several short-term and `one-off' studies. These engagements investigated the use of OurPlace as a seamless, place-based learning tool, garnering feedback as to how the application could be improved. The findings of these studies are presented and discussed through themes relating to seamless learning practices, engaging and empowering learners through control and content ownership, and the use of mobile learning technologies for civic engagement and inquiry. These findings were also used to assess if OurPlace met the design goals relating to learning in formal education contexts.

\textbf{Chapter \ref{chap:student-created}} explores the use of mobile learning platforms such as OurPlace in supporting project-based learning pedagogies, where students take the role of Activity designer and the application is used as a component within a larger project. The concept of `project-based mobile learning' (PBML) is introduced through the creation of a PBML framework, with this framework being applied in studies held with four different UK schools and a summer school of Travelling Showmen. This chapter covers the framework, suggestions for its configuration in response to contextual challenges, reflections on how PBML can harness students’ existing desires for independence, and how it could offer new avenues for leveraging place as a learning resource. 

\textbf{Chapter \ref{chap:discussionConclusion}} discusses the findings identified during the previously discussed studies, with discussions pertaining to how mobile learning technologies can be configured to support place-making, and recommendations as to how researchers and designers can better utilise the infrastructures of place as resources for civic mobile learning. I also discuss some of the study’s limitations, before concluding by responding to the research questions presented in Section \ref{sec:ResearchQuestions}.