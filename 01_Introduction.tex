\chapter{Introduction}

\epigraph{`An empathetic and compassionate understanding of the worlds beyond our own places may best be grounded in a love of a particular place to which I myself belong. In this way, we may recognize that what we need in our everyday world has parallels in the worlds of others.'}{\textit{Edward Relph}}

\section{Research Questions}

How can mobile learning technologies better surface and utilise the civic value of places and empower the communities which give them meaning?

\subsection*{How can existing place and community infrastructures be better utilised as resources for mobile learning?}

\subsection*{How can we design mobile technologies which promote civic learning?}

\subsection*{How can we design mobile technologies which empower place stakeholders?}

\section{Research Approach}
\label{sec:ResearchApproach}

DBR - Design-Based Research: Putting a Stake in the Ground

\subsection{Data Collection and Analysis}

A member of the research team was present during each of the app’s deployments, providing technical support as necessary and taking field notes. All interactions and data collection were approved by and conformed to the requirements of Newcastle University’s ethics committee. Audio recorded, semi-structured interviews were held with the adult participants throughout the study to understand their reasoning behind their Activity designs and their opinions of the technology. A thematic approach [4] to coding was performed across data from interviews, children’s creations within the app and observational notes. The resulting codes were reviewed and qualitatively analysed by the authors, and then grouped into candidate themes. These themes were summarised onto paper for discussion and validation before being finalised.

\section{Summary of Contributions}

\section{Document Structure}