\chapter{Space, Place and Infrastructure}
\label{chap:SpacePlaceInfrastructure}

This chapter presents how previous research has explored and examined the qualities of space and place, people's relationships to them and how these factors have influenced educational theory and practice. I also examine how the Human Computer Interaction (HCI) research community has approached this as a design space, particularly through the lens of the Digital Civics agenda \citep{Olivier2015}.

\section{Space and Place}
\label{sec:SpaceAndPlace}

As noted earlier, Tuan argues that space and place are two concepts which respectively describe the physical and metaphysical properties of a location \citep{Tuan1978}. While many of their contemporaries treated geography as a literal science of the physical properties of the Earth, humanist geographers such as Tuan and Relph \citep{Relph1976} positioned themselves within a more social framing, proposing the treatment of geography as more of a social science. Tuan argues that place is formed by our experiences with it: the more intimate a relationship an individual has with a space, the less abstract their cognition of it becomes. Harrison and Dourish similarly posit that while `space' is the three-dimensional structure of the world, `place' is an "understood reality" of mutually held and available cultural understandings of behaviour and action \citep{harrison1996}. They argue that while we are \textit{located} in space, we \textit{act} in place: the relationships we have with particular locations frame our behaviours within them. Furthermore, they argue that place can exist without an accompanying physical space: for example, through building behavioural expectations in virtual or augmented realities. They also posit that expected behaviours within places also change with time: be that time of day, or as larger trends change over months and years. Malpas argues that place, while related to space and time, is distinct from them, and is "methodologically and ontologically fundamental" \citep{malpas1999}. He proposes that place is tied to both subjectivity and objectivity, and that places become significant from the grounding of our experiences in them, rather than the experiences themselves. As such, he argues that \textit{who} we are reflects \textit{where} we are: while the land carries `a cultural memory and storehouse of ideas', our identities are also formed by the places we inhabit. This also raises the possibility of individuals seeing places as a part of their identity: a potential explanation for place stakeholders volunteering their time and resources towards a particular site's upkeep.

Relph takes a more phenomenological perspective, and argues that in order to be able to encourage the making of new places or the maintenance and restoration of new ones, we must first further understand how we experience both space and place and be able to describe what makes a place special \citep{Relph1976}. He argues that we experience place on a spectrum (varying from pragmatic and perceptual on one end and abstract thought on the other) in instances of various intensity throughout everyday life, often subconsciously, with each playing a part in how we experience space. On the more abstract side of this continuum, Relph introduces the concept of \textit{insideness}, which describes the relationship someone has with a place. He posits that if someone feels \textit{inside} a place they feel positively about it (e.g. safe, enclosed, at ease). \textit{Outsideness} describes the opposite (e.g. threatened, exposed, stressed), which would often occur when someone feels a division between themselves and their lived environment (feeling homesick, for example). Relph argues that the more `inside' of a place a person feels, the stronger their identity with it will be. Furthermore, he also claims that there are qualities and characteristics of place that can affect how we experience them: that environments which provide `genuine experiences' through direct, unmediated access to their social qualities and constructs (e.g. their local heritage) offer an \textit{`authentic'} sense of place. However, he argues that this is increasingly being overshadowed in our modern era by an attitude he refers to as \textit{`placelessness'}: \textit{`the casual eradication of distinctive places and the making of standardised landscapes that results from an insensitivity to the significance of place'}. Relph asserts that by undervaluing places' distinctive characteristics, being overly concerned with efficiency and accepting environments which are interchangeable to the point of anonymity, we run the risk of normalising \textit{`inauthentic'} experiences of place. The result is something superficial and contrived, and unlikely to represent communities authentically. However, this framing of `cookie-cutter' town planning as being `placeless' has received criticism for being condescending and elitist towards towns often lacking in social mobility, such as cities in Northern England \citep{Cresswell2016}. Relph also warns of the dangers of `museumisation'---the simplification and sanitisation of history to create a more palatable ideal. He argues that by highlighting only the best bits of local history, we run the risk of creating a `Disneyfied', inauthentic image of place. Smith criticises Relph for lacking a degree of perspective in this regard (e.g. not considering `\textit{the wonder of a child's eyes at Disneyland}'), and for not offering suggestions for how to go about actually developing `authentic place': rather, Relph simply argues that attempts to create place will only reinforce mass stereotypes of it \citep{smith1978}.

While I am partial to Relph's more phenomenological perspective, the criticisms regarding a potential lack of accounting for perspective ring true: if place is such a subjective, experiential concept, how could one be able to label another's neighbourhood as `placeless'? Ironically, this fits Relph's own words: `insensitivity to the significance of place'. Instead of being these spaces being placeless, it implies to me that the value of places and their \textit{authenticity} is merely less accessible to observers, potentially due to cultural and socio-economic differences between them the place's stakeholders. I believe that the implications of these arguments on HCI research is that when designing for or conducting research within a location, we must gain perspective on it: become more aware of the social constructs which underpin it, give it value to its stakeholders and create \textit{place}. To better understand these factors, it's likely that we as researchers will have to grow relationships with a place over time through direct habitation, or at least involve existing stakeholders as participants who have ongoing relationships with it. However, this becomes both more interesting and potentially problematic when accounting for Tuan's arguments of the influence of personal experience--because a person's place attachment is formed based upon individual experience, researchers should be aware that their perceptions of place may not align with others', running the risk of inappropriate design decisions if not handled with care. 

\section{Infrastructures of Place}
\label{sec:InfrastructuresOfPlace}
In order to design technologies which support the highlighting and sharing of place with other people, we first need an understanding of how place is encountered, experienced and understood. In her ethnographical study of physical infrastructure (such as the sewers and power supplies of cities, and the stairs and ramps of building entrances), Star argues that meaningful ethnographic study of these systems can open up an \textit{`ecological understanding'} of place \citep{Star1999}. She posits that infrastructure is \textit{`both relational and ecological---it means different things to different groups and is part of the balance of action, tools and the built environment, inseparable from them'}. In this regard, the humanist geographer's place and Star's infrastructure are similar in many ways. Star argues that the study of the physical infrastructure of sewerage, water and power supplies within cities can help one gain insights into distributional justice and planning power. One given example is that a keen observation of the usage (or omission) of stairs, ramps and railings can give an impression of institutional attitudes towards---and considerations of---people living with physical disabilities. I find that this argument holds merit, and it isn't hard to find real-world instances of it: for example, the staircase outside Vancouver Law Courts [Figure \ref{fig:stairs}] has been described by an accessibility blogger as \textit{`a serious disaster waiting to happen [and] dangerous for everyone not just those with disabilities'} \citep{Wheeler-Hall2017}. Another modern example of infrastructure reflecting an unequal distribution of power would be the water crisis in Flint, Michigan \citep{Clark2018}---the cause and response to which has been labelled as systemic environmental racism \citep{MichiganCivilRightsCommission2017a}. Star recalls that few participants in one of her projects utilised the final system that her team designed, despite the researchers following the principles of participatory design throughout the process. They identified that this was not because of usability issues with the interface, but rather how their design was a poor fit with the infrastructures the participants had to work with. The article highlights that the study of infrastructure in an ethnographic enquiry can uncover tacit conventions of everyday practices, allowing the unpacking of relationships between different communities, interest groups and perspectives.

\begin{figure*}
  \centering
  \includegraphics[width=0.8\columnwidth]{images/chapter02/stairs.jpg}
  \caption[Vancouver Law Court staircase]{The staircase outside of Vancouver Law Court, criticised for its lack of accessibility.}
  \label{fig:stairs}
\end{figure*}

Dourish and Bell argue that infrastructure doesn't just simply comprise of a space's physical properties, but also of different social, institutional and historical factors \citep{Dourish2007, Dourish2006}. They claim that these infrastructures are both embedded into social structures, whilst also serving as structuring mechanisms themselves. Infrastructures such as street names, regions, traffic flows and calls to prayer shape a person's experience by making it meaningful in different ways, but are themselves moulded over time into configurations which support social practice. Dourish and Bell argue that highlighting these infrastructures serves as a method to understand the social and cultural practices that occur within a space: the organisation of space becomes layers of infrastructure, through which we experience the world and produce, understand and enact cultural meaning. While the authors do not refer to the concept of `place', I posit that the combination of the arguments by Star and Dourish and Bell build upon and solidify Relph's notion that we have various forms of encounters with place in our everyday lives: that these infrastructures are the fundamental elements through which we encounter \textit{space} and form \textit{place}. In line with Relph's argument that our encounters with place take can be placed on a spectrum from the perceptible to the abstract, I argue that these encounters can be seen to be with various forms of place infrastructure (be that literal or social), created by our actions within space and place---such as the building of physical infrastructure or the enactment of daily rituals.

Dourish and Bell conclude with some implications for design: they argue that because space is organized both culturally as well as physically, a cultural understanding of a place can provide meaningful and coherent framing to relate it to human activities. In the context of HCI, the technologies we design can act as infrastructure within a place, with each having varied social and cultural interpretations and meanings. They also argue that technology designers need to be aware that both space and place feature both physical and social boundaries and transitions, and that these are not always things that technology should try to bridge. They claim these boundaries are frequently used by inhabitants as an asset, and that technologies which introduce `seamlessness' can detract from a place's value. Finally, Dourish and Bell argue that because a) new technology can introduce new layers of infrastructure and b) our encounters with places are formed through their layers of infrastructure, introducing new technologies to a location can inherently cause people to `re-encounter' and re-evaluate it as a space and place. In summary: technology can---as a layer of social infrastructure---act as a destabilizing, transformative force in how we experience place, and can affect stakeholders' experiences with place in different ways.

\section{Place-making}

Encounters with these social, cultural and economic infrastructures are especially important for forming relationships with space---a process frequently called `place-making'. This section will cover some of the ways that researchers have approached place-making, including how it occurs; how existing relationships with place can be surfaced or articulated; and how technologies can be configured to support place-making processes.

\subsection{Building Relationships with Place}
While Tuan takes a passive outlook on the creation of place (where our relationships build through experience over time), Harrison and Dourish give a more active and designerly perspective: they argue that `\textit{Space is the opportunity; place is the understood reality}' \citep{harrison1996}. In other words, they posit that our relationships with place are things that can be encouraged through design. It also implies a form of learning: place forms through the development of an \textit{understanding} of reality. These understandings of place could even be up for differing interpretations---after all, an understanding is based on an individual's perspective. For the purposes of this project, I want to explore how mobile learning technologies might utilise these opportunities afforded by space as learning resources, and how place-making can be utilised as a mobile learning experience.

Relph describes a number of ways in which a person's relationship with a new place could form. Building on his concepts of \textit{in/outsideness}, he introduces the concept of \textit{vicarious} insideness---that which occurs when someone engages with a place through their imagination (e.g. through experiencing a place in works of art or reading about it in a book) \citep{Relph2018}. He claims it is most pronounced when a place's depiction corresponds with our own experiences of places we are familiar with. Relph notes another, more deliberate form of place-making called \textit{empathetic} insideness, which is a very deliberate attempt by an individual to understand a place in depth. This requires a \textit{`willingness to be open to the significances of place, [with] the hope to see it as rich in meaning for those whose place it is'}. These factors can coalesce into authentic and self-conscious place-making, where \textit{`there is sensitivity to the significance of place in everyday life'}, and can be encountered in instances where communities and individuals have invested hopes and ideals in actively making a place for themselves. Educational technologies may be able to use these types of place-making, providing opportunities for people to develop relationships with place through learning experiences. Mobile learning technologies could be particularly effective at this, as they could offer experiences which support the development of empathetic insideness within the authentic learning context, without a reliance on doing this remotely (requiring the learner to be open to vicarious insideness).

In their review of environmental psychology-based sense-of-place literature, Kudryavtsev et al. note that many researchers suggest that a `sense of place' is a combination of two complementary concepts: \textit{place attachment} and \textit{place meaning} \citep{Kudryavtsev2012}. \textit{Place attachment} refers to the bond between people and places---the degree to how much an individual or people value or identify with a particular place. This includes the extent to which a place satisfies an individual's personal requirements, as well as how a place becomes a part of an individual's personal identity and---at least partly---defines them as a person. In this way, place attachment fits with Malpas' outlook on our relationship with place: that our identities are formed by the places we inhabit, and that \textit{who} we are reflects \textit{where} we are \citep{malpas1999}. Kudryavtsev et al. argue that \textit{place meaning}, however, refers to something closer to the humanist geographers' interpretations of place: the meanings that individuals ascribe to settings that they are familiar with, reflecting their environment, social interactions, culture, politics, economics and history. They note that within the context of environmental education, relatively little research has been done on the combined effects of place attachment and place meaning on behavioural change. However, they hypothesize that a combination of both factors would be more effective at fostering place-based behaviours (in this case, pro-environmental actions) than either taken separately. From examinations of previous works, they suggest that place attachment which hasn't placed an emphasis on ecological elements may not necessarily contribute to pro-environmental behaviour---they suggest that also introducing a pro-environmental place meaning could foster this behaviour more effectively. One might also speculate that this could be applied to many other infrastructural elements of place, outside of the topic of environmentalism. For example, someone might not concern themselves with the socio-economic issues surrounding funding the maintenance of a local park, even if they have built a `place attachment' through a dependence on it for walking their dog. Fostering this element within someone's place identity would likely involve introducing them to relevant infrastructure and forming new place meanings. This project will explore the potential for mobile learning technologies in developing (or demonstrating others') place meaning and place attachment, highlighting both as opportunities for learning.  

\subsection{Technology as a Mediator for Place-Making}
\label{sec:TechMediator}
Relph argues that one of the primary ways to build a relationship with space is through having experiences with it, and notes that the majority of modern experiences of landscapes are mediated by machines \citep{Relph1976}. Having written his book in 1976, the dominant machine of the time was the automobile. He posits that while at the time there was a narrative that cars had separated people from landscapes and places, this was reductive---the availability of the automobile also extended people's mobility and fundamentally changed how they experienced the world by allowing for new options, comforts and experiences with places that otherwise would likely be inaccessible. Similar arguments can be levied for and against the dominant machine of the 201X's---the smartphone. A common narrative is that digital technologies separate us from `the real world' through distraction, and this may be true in many cases. For example, an over-reliance on technology for navigation (e.g. Google Maps) has led many to repeatedly use the same predetermined routes through their local environment, reducing their exploration and limiting opportunities for building relationships with new places \citep{Lochtefeld2019}. But, similarly to the automobile, the increasing ubiquity of mobile technologies has also dramatically increased users' access to information and experiences that had previously been inaccessible. Modern mobile devices act as a portal to places (both physical and virtual) and their stakeholders all over the world. As such, mobile learning technologies have an opportunity to highlight these stakeholders and their knowledge and values, both in-situ and remotely---allowing learners access to new interpretations of place.

Harrison and Dourish posit that as place is created by patterns of use, `placeness' is not something that we can design technologies \textit{in} \citep{harrison1996}. They argue, however, that we can support placeness by designing \textit{for} it: creating new spaces (or augmenting existing ones) within which people can make place. They argue that technologies can support the emergence of a sense of place through supporting adaptation and appropriation: allowing for individuals to re-arrange elements of (physical or virtual) space to suit and reflect their lives and sense of self. Extending this to the domain of mobile learning, it would make sense that sharing these arrangements and self-reflections would be a way of sharing one's own interpretations and understanding of place (for example, community heritage or personal history) and supporting vicarious insideness with others.

Giaccardi et al. note that this new prevalence of digital technologies has opened the door for new ways of interacting with heritage \citep{Giaccardi2008}. They claim that by using `cross-media interaction' (the use of multiple forms of both media and technology) to create new socio-technical infrastructures, novel interactions can be enabled between local communities and their places of heritage within authentic environmental settings. These infrastructures allow for new cultural experiences, articulating and exploring existing people's relationships with place in new ways. I would argue that these experiences open up new opportunities for visitors to build place attachment through empathetic, vicarious insideness---supporting the creation of `placeness' as suggested by Harrison and Dourish. Giaccardi et al. similarly argue that sustaining this knowledge of social relations is a place-making process, which technology can support by supplying communication and interaction spaces in which communities can engage with the physical and social settings of heritage. They argue that little HCI design had been done with the aim of reinforcing (or recovering if lost) the relationships between communities and their places of meaning, offering that one possible `solution' could be cross-media interactions. These interactions would allow people to \textit{`express their perceptions, interpretations and expectations about the heritage'}, reinforcing a sense of place through repeated interactions over time. They note that of particular importance is making heritage a `living practice': \textit{`giving people active and supportive roles, [engaging] them in connecting to each others' experiences, considering each other's interpretations, and building insights that may lead to new meanings and relationships'}. Additionally, Giaccardi et al. suggest that the use of technical infrastructure alone is not enough to support heritage practice and place-making---they argue that social infrastructure is necessary for the support and regulation of community participation over time. They also suggest that designs in this space should legitimise personal accounts as far as possible, in order to encourage the collection and conservation of resources---many of which are likely to have unexpected value. This is also linked to another important factor that the authors recognise: the participants' sense of ownership over the content, which strengthens their relationship with the heritage and places of meaning.

Highlighting the distinction between space and place, technology can also support `authentic', personal experiences with place without the need to be at the same geographic location. For example, Google Earth VR allows users to explore most locations in the world from a first-person perspective, using a combination of satellite imagery, street-level photography, and machine learning to generate recognisable 3D environments \citep{Google}. Taken as presented, the software is more reminiscent of \textit{space} rather than \textit{place}---the software does not attempt ascribe any emotional connections or value to particular places, outside of featured landmarks (e.g. the Eiffel Tower). 

\begin{figure*}
  \centering
  \includegraphics[width=0.8\columnwidth]{images/chapter02/googleEarth.PNG}
  \caption[Jeff Gerstmann showing Google Earth VR]{Jeff Gerstmann shows his colleagues his childhood school in Google Earth VR.}
  \label{fig:googleEarth}
\end{figure*}

However, during a live-streamed internet show, video game critic Jeff Gerstmann demonstrated the software, claiming that the sense of place offered by the experience was \textit{`profound, and almost emotional'} \citep{Gerstmann2016}. Rather than explore far off locations which would be otherwise inaccessible or the featured landmarks which are of worldwide renown, Gerstmann opted to explore places he already had relationships with: \textit{`Last night I just took my whole commute home, stood in my own backyard.'} [Figure \ref{fig:googleEarth}]. In this instance Gerstmann used Google Earth VR to highlight his place identity, sharing his relationship with place with thousands of others, allowing them to vicariously encounter it through the Internet. Examples such as this highlight the potential for technologies to allow the sharing of different interpretations of space meaning, by giving stakeholders the means to share their past experiences, knowledge and insights to encourage vicarious insideness.

\subsection{Place-Making in HCI Research}
\label{sec:PlaceMakingHCI}

The HCI research community has been investigating this potential for place-making through technology for some time. McCarthy and Wright posit that place-making can be viewed as a dialogical process, in which a person's relationship to a place develops over time through repeated relational interaction and interpretation \citep{McCarthy2005}. They argue that this is a two-way `conversation-like' relationship between the person and their environment, with both contributing qualities which together build a relationship. For example, this might include the person's sensory experiences with the environment, the socio-cultural history attached to it and even the outlook of possible future engagements between the two. The authors suggest that for technologies to help people feel `in place', they should engage at a personal level, rather than treat them as an anonymous entity. 

A good demonstration of many of these concepts in practice can be seen in a study by Crivellaro et al., who worked with multiple heterogeneous stakeholders in the context of a housing estate undergoing urban regeneration \citep{Crivellaro2016}. Through a series of engagements with current and former residents, the research team designed walking trails through the area and used a technology probe to collect participants' reflections. Walking trails were chosen as they were \textit{`seen as a means of encouraging a genuine engagement with the environment and stimulated pause and reflection'}. The engagement was configured to encourage the participants to convey what they valued about the estate and took a slower pace, facilitating \textit{`organic growth'} and the participants developing a sense of ownership of the trail. The participants were keen to ensure that the trail represented their estate both fairly and accurately---while the area had suffered from negative stigmas which they viewed as being unjust, they were also wary of portraying it in an unrealistically positive manner. This mirrors Relph's concerns around the `Disneyfication' of heritage: glossing over any negative aspects to make the consumption of it more appealing to a modern audience \citep{Relph2018}. Furthermore, the residents viewed their creation of the trail and audio logs as a form of both `anticipatory archaeology' (by documenting the regeneration process) and social curation: I posit that this can be looked at as them using the probe to safeguard their sense of place from being expunged during the estate's redevelopment, decoupling their interpretation of place from space through the use of a socio-technical infrastructure. This wasn't just for their own reflection on the past, either: these memories were recorded so that others could listen to them, allowing the stakeholders to take roles in other people's place-making with the estate. As the authors note: \textit{`the importance of "giving something back" points to the residents' desire to find value in their stories and actions, and see their contributions as having a wider and lasting impact.'} The participants were re-constructing the place's identity by using the stories of those who contributed to it. This can be seen to support McCarthy and Wright's positioning of place-making as a dialogical process \citep{McCarthy2005}: the participants are attempting to continue a conversation regarding place. Potentially as a way to make up for `losing' the estate as she knew it, the participant was creating further dialogue with future residents about her experiences of place. The engagements were designed to highlight the heterogeny of the stakeholders' experiences and opinions, while using the estate as a `common ground' with which all of the participants were familiar. The authors argue that collecting a diverse set of accounts that \textit{`enact place over time'} can open a space for more genuine portrayals of community. As McCarthy and Wright argued, the relationships between the estate and the stakeholders were seen to be built up on past experiences, the current situation and the outlook for possible future engagements. The participating stakeholders often had their own motivations and agendas (e.g. showing that life on the estate was generally more positive than its reputation would have suggested), surfaced by the probes thanks to their engagement on a personal level, recognising the participants as individuals. Such examples highlight the potential for technologies---particularly mobile ones, which can be taken to relevant physical contexts---for collecting, sharing and highlighting the differences between various stakeholders' place attachments and place meanings.

McCarthy and Wright also note that mobile devices are particularly well suited to engaging people on this personal level: by their nature, phones are intrinsically personal devices, and as the authors argue \textit{`allow people to capture the intimacy of interpersonal relations while moving from one place to another in a public sphere, blurring the traditional boundaries between public and private, intimate and extraneous'} \citep{McCarthy2005}. As an example, the authors cite \textit{RIOT!1831} \citep{Blythe2006}: an interactive play which used mobile technology to connect participants with the past version of their environment. McCarthy and Wright claim that the participants enjoyed having a private experience in a public place, afforded by the nature of the phones' handheld form-factor. They also note the participants' appreciating being in the authentic environment, with them valuing being within the `set' of the play. This concept could also be applied to Gerstmann's experiences with Google Earth VR, which elicited emotional reactions to visiting places from his life experiences in an `authentic' (if virtual) place. Interestingly, Google Earth VR was also similar to \textit{RIOT!1831} in that it allowed for private exploration and reflection in digitised versions of public places (e.g. being able to privately explore public streets which would ordinarily be crowded).

CrowdMemo was another project which used mobile technologies as a part of the place-making process \citep{Balestrini2014}. During the study, school children in Santa Fe, Argentina created video micro-documentaries using smartphones. These documentaries were about places and events important to the local community, and comprised of interviews featuring elderly people sharing their memories with the students. The documentaries were made available online, and accessible by scanning QR codes printed on commemorative plaques installed at places featured in the videos. From a place-making perspective, the researchers noted that the memories collected by the students were \textit{`imbued with features of the local identity, and publicly displaying them led to reflection on locations in the town and why they are relevant to the community's heritage'}. One of the participants claimed that the created documentaries being personal and relatable (and arguably, as a result, more authentic) helped promote reflection: \textit{`it's not about some texts and paragraphs put together by a historian, it's about the testimony of those who gave life to many of the situations in our heritage.'} The place-making impact of the project for local places of extended beyond the members of the immediate community, too: the school's headmaster noted that visitors to the town ask about the QR codes, and the community members use them to promote their culture at all times. 

Balestrini et al. conclude the paper with some recommendations for HCI researchers conducting community technology projects, some of which may be helpful within a `HCI for place-making' context. They recommend following action research principles: involving community stakeholders in the conception and running of the intervention and ensuring that it provides some value to each stakeholder. They argue that these were key factors in promoting a sense of community ownership over the project, with their stakeholders involved from the outset (to the point where the participants actually initiated the collaboration with the researchers). However, the authors note that a sense of ownership of the project would not have been enough to sustain engagement with it---they argue that this was achieved by providing value to all of the involved stakeholders (i.e. valuing the elderly participants' life experiences, giving the young students new technology skills, supporting the teachers in running an innovative educational project). Additionally, they recommend using existing, off-the-shelf technologies in novel ways, rather than introducing new, novel technologies. They note that many of their students already owned smartphones and therefore knew how to operate them, and were excited to be able to use these familiar devices in new contexts with a new set of skills. Reducing the skills barrier and the amount of new technical infrastructure required to participate helped make the project more sustainable. Balestrini et al. also suggest facilitating a range of face-to-face social encounters can lead to discussion and ongoing engagement. A given example is that the encounters between the children and the elderly participants was recognised as one of the most important elements of the project, as it meant that the elderly participants knew that their life stories were being valued. This value of face-to-face encounters is echoed in Crivellaro et al's study, for which the technology probe's integration into the project was designed to encourage face-to-face interactions between participants with different life experiences.

Through the results of their `Community Historians' project, Fox and Le Dantec explored how participatory technology design undertaken with communities could be better configured to support civic engagement and community empowerment \citep{Fox2014}. They found that their initial approach was more in their interests as researchers than in the interests of the participants, who had been marginalised and particularly disillusioned with academia. In response, the researchers re-framed their workshops to be more clearly and immediately advantageous to the community members that they were working with. This involved both backing away from immediately pursuing their research goals and meeting with community leaders in order to identify ways in which the project could benefit the community members. In response to the community's negative past experiences with academic institutions and their concerns that they were once again being reduced to simple objects of study, the researchers even stopped referring to them as `participants': instead, they were highlighted as collaborators through the use of the term `Community Historians'. With input from the community leadership, the researchers held a series of design workshops exploring how the use and creation of technology could empower residents in the articulation and performance of community identity. The workshops featured `Critical Making', where non-expert workshop participants had the opportunity to take a hand in the creation of device hardware from `raw' components (e.g. camera sensors, motherboards). This was done as a means of promoting reflection about the potential usage of technology within the community, with the authors arguing that a DIY approach can give non-experts a fast-track to unpacking ideas for the potential uses of technology. 

Fox and Le Dantec showed the Community Historians how to make portable cameras which automatically captured images upon detecting movement. However, some were uncomfortable using the cameras, as they were similar in function to the surveillance cameras used by the local authorities (who were seen as not having the community's interests at heart). This clash of researcher expectations with participants' reality is reminiscent of one of Star's projects (\cite{Star1999}, discussed in \ref{sec:InfrastructuresOfPlace}), where the project's solution was a poor match for existing infrastructures in the participants' workplace. The Community Historians project demonstrates that this can also occur with socio-economic infrastructures, not just physical and digital ones. This project was still successful thanks to the goals of the design process: rather than be `product/technology-focused' encounters, the workshops explored what those encounters would have aimed to achieve and how those goals could be accomplished. Rather than focussing on an end product, they worked towards identifying and developing processes to support community practices and empower them in \textit{`in the face of authority and power differentials'}. The authors reflect that HCI design interventions in community contexts need to respect and engage the community on its own terms, acting as a \textit{`balance against trends of rationalization and a rhetoric of disruption that underpin reductive moves to treat all communities the same.'} This argument falls in line with Relph's concerns about `placelessness' (\cite{Relph1976}, discussed in \ref{sec:SpaceAndPlace}), where he argues that by treating places (and, we can intuit, the communities which form around them and give them meaning) interchangeably or at too large of a scale results in `inauthentic' experiences of place. Having a false or incomplete picture of a place's infrastructures can lead to inappropriate design decisions, as seen in Star's unsuccessful project. Fox and Le Dantec present a convincing argument for the fundamental advantages of truly participatory design: involve and emphasize the agency and perspective of community members from the outset, as they are the ones best positioned to inform the design process. As the authors posit: \textit{`a mode of intervention that is based in community practice shifts the power to the community, so that it is not technology and data usurping local influence and ability, but instead technology and data selected in ways to support, preserve and amplify local influence and ability'}. In short, the researchers found that forming partnerships with communities and co-developing alongside them can be an effective way to encourage the articulation of an authentic shared community identity. 

\section{Digital Civics and the Spatial Citizen}
\label{sec:DigitalCivics}
Research held as part of the Digital Civics agenda frequently engages with the socio-economic relationships between communities and place. This section will give a brief synopsis of the Digital Civics agenda; how various Digital Civics projects have engaged with space, place and citizenship; and the concept of the `spatial citizen'.

\subsection{The Digital Civics Research Agenda}

As a result of the economic crisis and resulting austerity measures enacted by the UK government over the last decade, many local authorities have been forced to implement severe cuts to their public services (including---but not limited to---waste management, transport, parks and recreation, education and social care). Olivier and Wright developed the Digital Civics research agenda at Culture Lab (later renamed to \textit{Open Lab}, where this research took place) as a direct response to these developments, claiming that as a research group in a civic university (one which is \textit{`embedded in, and responsive to, its local context'}) they were `compelled' to reflect on how their HCI research could be of use and value to the local authorities and citizens \citep{Olivier2015}. Prior to the Digital Civics agenda, Culture Lab's research had been human-centred and participatory, providing systems and services which were both meaningful and helpful. However, they reflected that their work had been detached from the local context---the research often \textit{`failed to extend beyond the confines'} of their projects, meaning that it frequently could have been done anywhere. They also realised that they were only working within (and, as a result, proliferating) the status quo of service delivery from institution to citizens: they were giving people some input on the design of products, but in a way which still supported the framing of public services as being something `done to' citizens without providing any alternative models. Digital Civics moves away from framing citizens as consumers and towards a model where citizens can take an active role within participatory systems, thanks to new forms of relationships between citizens, businesses and local authorities. Olivier and Wright admit that meaningful, systemic change such as this will take significant amounts of time. Even within the smaller scope of research projects, they posit that the development of long-term relationships between researchers, citizens and local authorities will be necessary if new relational models are to be realised and the potential roles for technology within them discovered.

\subsubsection{On the Dangers of Libertarianism and the Impacts of Big Society}
Before continuing, it is worth mentioning that Olivier and Wright note that there is also a danger of the Digital Civics agenda being warped or misconstrued as \textit{`finding ways of making citizens do it for themselves, or dismantling public service provision'}. Digital Civics was imagined in the context of a period of austerity. As a part of this, many changes were put in place by the UK's conservative government under the guise of localism---part of David Cameron's `Big Society' initiative which purportedly aimed to give local authorities the power to undertake local solutions to local problems, rather than continue to centralise power in Parliament. This agenda was ratified in the Localism Act of 2011, which de-regulated and/or removed many of the constraints related to local issues of housing and taxes \citep{MinistryofHousingCommunities&LocalGovernment2011}, and coincided with a number of austerity measures put onto public services and placing greater emphasis on volunteerism. While the principle of de-centralisation was seen as agreeable across much of the political spectrum, the `Big Society' approach was met with public scepticism. Polls found that over half of respondents thought that the Big Society measures were `just an excuse' to save money by cutting public services, and that only around 10\% thought that Big Society would be a success \citep{Ferragina2017a}. Furthermore, while the restructuring put in place by localism measures relied on more pro-active and engaged citizenship from the public, some argued that not enough resources were allocated to supporting this citizenship actually occurring. As Rogers argued at the time: \textit{`Most of the political problems [the Prime Minister] faces, from cutting crime to reducing obesity, can only be met if residents and citizens play their part. Yet the state has so far invested very little in teaching the skills that could help people make a contribution'} \citep{BenRogers2010a}. This lack of support meant that citizens who wanted to take advantage of the powers given by the Localism Act in areas such as town planning had to invest considerable time and effort, as their output was to be judged to the same level as professionals \citep{BBCSundayPolitics2013}. These expectations of large amounts of free time for research and volunteering would exclude many from the empowerment promised by the legislation, particularly those who had already been most impacted by cuts to social services and were likely to be time-poor.

It is within this context of volunteerism in the stead of well-funded public services that Digital Civics must walk a thin line: between supporting citizens living in the results of austerity and supporting the austerity measures themselves. While in some cases there may be a danger of Digital Civics projects being seen to re-configure the services provided by local authorities to enable a `small government' model, this is is not the intention of the agenda (at least, as I have read it). This libertarian approach (in the contemporary and primarily American sense) is completely contrary to the motivations behind starting the research agenda in the first place: mitigating the damage done by conservative austerity politics upon public services. Instead, Digital Civics projects should aim to strengthen relationships between citizens and local service providers: instead of reducing the role government has within the lives of each citizen and relying on a `DIY' approach, it should aim to empower citizens to have more involvement and agency within their government's processes. This key distinction means that rather than designing in preparation for the permanent loss of public services, Digital Civics technologies should work to mitigate hardships inflicted by austerity measures in a way which also implements improvements for when these measures eventually come to an end. 

\subsection{Digital Civics and Place}

There have already been several projects within the scope of the Digital Civics agenda which are related to the use of technologies within space and, more importantly, place. While each of these projects addressed this agenda, each took a different approach to highlighting and utilising elements of place.

\subsubsection{PosterVote}
While the PosterVote project \citep{Vlachokyriakos2014} predates the publishing of Olivier and Wright's declaration of the Digital Civics agenda \citep{Olivier2015}, it nonetheless fits within it thematically. The project explores how low-cost technologies could be utilised by communities to support grassroots democracy and social action. The PosterVote system consists of a low cost, lightweight piece of hardware, stuck onto the back of a piece of paper. A question and up to five responses is printed onto the paper, with each response having one of the hardware's buttons underneath. The system simply records users' choices, which can be reported back through a machine-readable series of LED flashes and beeps. 

The motivation behind the project stemmed from how most uses of technology for civic engagement (e.g. showing discussions on public displays) frequently require technical knowledge and are `mostly initiated or managed by local political organizations and local councils'. As a result, these institutions are still the ones driving agendas, and usually only using these technologies as consultation tools to increase perceptions of efficacy. Vlachokyriakos argues that the high cost and top-down nature of these systems make them `inappropriate for activism'. Similarities can be drawn between these existing civic engagement technologies and mobile learning---creating new, bespoke mobile learning technologies and experiences can require significant technical knowledge, and commissioning the creation of one frequently incurs significant costs. As a result, bespoke mobile learning technologies are largely out of reach for most community-driven organisations.

Vlachokyriakos et al. proposed a technological solution, designing PosterVote to support the diverse viewpoints of activists and stakeholders by removing the need for technical skill and significant funding. Furthermore, the nature of the design harnesses different engagement levels of people within communities: individuals who are the most engaged and are willing to put more effort into a project may choose to set up their own PosterVote instance, while less engaged members can simply use the devices to share their views or make use of the data others have collected. This kind of approach (which is frequently used for community-led projects such as Wikipedia, although concerns have been raised around its sustainability \citep{Simonite2013}) is already how many groups of place volunteers function: more engaged members helping to maintain a place, whilst less engaged members of the community take a more `consumer' role. As such, it's easy to imagine a similar approach also working with regards to digital representations of place through mobile learning technologies.

The authors note that the low cost of the posters initiated discussions about their ownership: some participant groups treated the deployments as an effort to be owned by the community (without making any real distinction between the posters' organizers and the wider community), while others were cautious about democratising the process too far and kept a more rigid hierarchy. Such questions of ownership are likely to be raised in this project too, particularly in regards to wider communities being able to share their interpretations of place, which may clash with the views of the owner/controlling body of the corresponding space.

Vlachokyriakos et al. also report that one of PosterVote's main advantages is that is can operate within relevant space and place---the nature of the technology means that it can be placed in a location relevant to the question being asked, and that people's participation with the system can be configured according to where and how it is deployed. Participants noted: `\textit{The thing about having it on a lamppost is it's directly relevant to that particular position.}'. While there is likely a value in such a technology being static in place (e.g. a concrete association between it and the space in which it resides), mobile technologies would also have the advantage of being able to traverse between different contexts, allowing for similar levels of relevancy to multiple places. 

\subsubsection{FeedFinder and App Movement}

This potential of mobile technologies to be used for users to share their opinions of place has been explored in two other Digital Civics projects: FeedFinder and App Movement. 

Designed as a response to a perceived lack of practical and moral support for breastfeeding in public spaces, FeedFinder is a smartphone application designed to support breastfeeding women in finding, reviewing and sharing public breastfeeding-friendly places \citep{Balaam2015}. The application allows users to add locations such as businesses to a map and review them based on relevant categories. Reviews are public, meaning that others can see how places have been rated to make more informed decisions about where they choose to breastfeed. Some women even used the system to try and effect change: for example, one participant showed the application to a department store's manager, comparing them to a competitor's ratings as a way to get them to improve their facilities. 

The authors discussed a tension between user expectations of available content and the platform being entirely reliant on grassroots, community-generated data. For many users, the value of FeedFinder was dependent on having a significant number of reviews available. As one of the study's participant noted: `\textit{I love the idea, but there's no places listed! [...] You shouldn't just rely on user submissions as people won't use an app with no content.}' This highlights a potential issue for platforms designed for `browsing' places---the value of FeedFinder was largely in the ability to peruse places, and so it suffered without having a significant selection. While PosterVote relied upon people designing and running their own deployments, the value of the platform could be assessed on a per-deployment basis: having hundreds of different posters in a region would not meaningfully affect the user experience or value of the platform to its end users. 

Despite this, FeedFinder served as a tool that facilitated the collection and sharing of `\textit{lived experiences}' of breastfeeding in public, and the comparison of these experiences on a local, regional and national level. This supported participants in not only finding more comfortable places to breastfeed, but also provided a way to compare lived experiences to the presumed rationality: that the public is not supportive of breastfeeding in public. These surfaced lived experiences acted as evidence for publics who wanted to affect civic action and real social change. This also highlights another possible area for investigation in this research project: how place-based mobile learning technologies could be used as a method for civic action by stakeholders as a part of meeting their own agendas.

Following on from FeedFinder, AppMovement is a platform which enables the promotion, design, production and deployment of community-commissioned mobile applications \citep{Garbett2016}. Using a website, users are able to propose a idea for a location-based review mobile app (e.g. "Safe places to fly your drone"), each functionally similar to FeedFinder. Communities can commission similar applications, bespoke to their own contexts and requirements. As well as the place review data within each app being community generated, the applications themselves are proposed, produced and promoted by their own community of interest. As such, App Movement extends the `grassroots', community-contributed nature of FeedFinder into the production of the application itself. 

Garbett et al. note that many communities of interest had already demonstrated that they're capable of (re)appropriating technologies for their own purposes, but issues of cost and technical know-how frequently prevent them creating their own bespoke solutions. As noted earlier, the same can be said for volunteer-based organisations caring for community spaces. In response to this, App Movement makes an effort to democratise the creation of mobile software: serving as a blend between PosterVote and FeedFinder, it lowers the cost and technical requirements of producing technologies which can be used by communities for their own requirements, independent of top-down institutions. As touched on by Vlachokyriakos et al. in the PosterVote project, the authors note that this `community DIY' approach led to a sense of ownership and stronger engagement: `\textit{Proposing an idea leads to a sense of ownership of it. The result of this sense of ownership is the increased motivation to promote the concept and engage the community in the appraisal of the idea.}'. Furthermore, they argue that the democratising of the app creation process opens it up for appropriating by communities, allowing them to `\textit{more accurately address [the] issues they face}'. These projects highlight the potential for place-based technologies to foster a strong engagement with communities by giving them approachable creation tools: technologies which give enough creative freedom to support communities in addressing the issues they care most about.

\subsubsection{WheelieMap}

WheelieMap is another Digital Civics project which explores how digital space-based technologies can support civic advocacy \citep{Kirkham2017}. The platform is designed to support wheelchair users in identifying, documenting and reporting areas which have accessibility issues by recording and uploading a combination of motion data, video clips and GPS location data. When combined with qualitative user reports, the system can empower wheelchair users to map inaccessibility and advocate for improvements. This approach improves on existing solutions, which frequently rely on expert documentation (which is expensive), purely automatic systems (which lack qualitative assessments) or `offline' community efforts (which frequently lack actionable evidence for decision makers). 

When reflecting on their experiences with the system, some participants noted that it offered a potential for sharing their point of view and assisting in empathy with the wider public: `\textit{I think it would be really good to show the general public things they otherwise wouldn't think of [...] if you don't experience it yourself, you just don't think about it sometimes. This terrain would be really easy to walk along; you don't realise what it would be like for someone in a wheelchair or with a pram.}' In this context, the technology was used as a tool to assist in communicating to others the participant's interpretation of place (particularly the infrastructures within it, such as paths), and how it could be improve. Unlike App Movement, WheelieMap is limited to the single context of accessibility. However, it acts as an example to how place-based mobile technologies can highlight others' experiences and interpretations of place, and how such submissions could be used as educational resources and for civic action.

\subsubsection{Data:In Place}
With the amount of data being generated regarding places and people increasing daily thanks to developments such as smart cities, questions have risen regarding how such data could be made openly accessible for use by citizens. Data:In Place is a web platform designed to support the open access and sense-making of data for the purpose of civic advocacy by citizens, enabling effective action in relation to place-based issues and concerns backed by relevant data \citep{Puussaar2018}. Rather than focusing on community-generated content as with the previously discussed projects, Data:In Place instead explored how interpretations of place could be explored or evidenced through providing easy access to existing data.

The authors worked with a group of residents interested in starting a Neighbourhood Plan, a process introduced through the Localism Act as discussed earlier. The Data:In Place platform supported these citizens in using data as evidence for their civic action. Previously there had been technical and knowledge barriers in place which had distanced them from using it effectively, relying on third-party professionals and raising issues around dependency, economic exclusion and misrepresentation. The authors argue that being able to easily access data through the platform also supported participants in exploring local issues and more deeply understanding their communities. While Data:In Place takes a much more quantitative, data-driven approach to discussions around place than I am interested in exploring, it does demonstrate how digital technologies can be used to gain and share deeper understandings of local issues and interpretations of place.

\subsubsection{ThoughtCloud}

ThoughtCloud is a feedback system, designed to be deployed in situ where voluntary and community care organisations deliver their services \citep{Dow2016}. The system supports both quantitative and qualitative feedback, through the use of Likert scales and the recording of video and audio messages. Suggestions for feedback topics can be configured by the event organisers by them supplying a set of questions. ThoughtCloud was motivated in part by the fact that stakeholders who use and rely on certain services are frequently under or misrepresented in existing feedback pipelines, with reasons ranging from stigmatisms of particular services to tokenism.

One of the study's findings was that while some of the more structured responses were of limited use (due to the participant being overtly guided by a third party or a leading question), some of the more unstructured qualitative data was seen to provide `\textit{richer accounts of personal experience of the provided services, and how people saw themselves as members of a community.}' This suggests that participant-led free-form audio and video recording could be a good medium for gaining insights into stakeholders' relationships with the social infrastructures they engage with, and how they position themselves within a community.

\subsubsection{Community Conversational}

Community Conversational \citep{Johnson2017} is a Digital Civics project with direct ties to the issues surrounding the UK government's localism measures discussed earlier. The project focuses on community organisations' and local authorities' consultation engagements with local residents. These organisations had a responsibility to involve local residents in consultations and provide evidence for both the fact that these engagements took place, and that the views and opinions raised by residents were being taken into account in the decision making process. In keeping with the increasing reliance of volunteerism due to cuts to local authorities' funding, running these engagements had become the responsibility of volunteer-based organisations. The researchers identified that these consultations frequently failed to capture rich insights from participants due to numerous issues, including volunteer groups lacking the resources and research experience necessary to effectively capture and analyse data. In response, Johnson et al. produced Community Conversational---a workshop activity which took the form of a board game, augmented with video recording and an online data repository. Game pieces could be placed on a map, with the system tracking their placements and matching them with recorded audio.

While the participants valued the more open nature of the conversation---thanks to it not `\textit{being bound by and driven by council officers}'---the decision making facilitators struggled to make use of the more more open qualitative data. The researchers noted that despite the collected data being `rich', the decision makers saw it as being of limited use, having previously aimed to collect quantitative results as evidence of support for previously identified solutions to a given issue. The authors argue that this points towards the organisations recording the opinions of local experts as a tokenistic series of bureaucratic tick boxes, rather than including them in meaningful consultations. This was further evidenced by the council representatives using the data purely in relation to predefined issues, limiting the practical value of a rich data set. Comparing the findings of Community Conversational with those of WheelieMap and ThoughtCloud makes it clear that even if consultation participants are contributing rich insights about their use of space, the value institutions take from participants' shared interpretations of place is limited by the way they are analysed and responded to. I would argue that such findings highlight the need for platforms outside of such institutional control, through which stakeholders are able to share their values and concerns regarding place with others without the need to go through institutional filters.

\subsubsection{Gabber \& TalkFutures}
Gabber explicitly aims to tackle this issue by supporting stakeholders in contributing directly towards the collection and analysis of qualitative data \citep{Rainey2019}. The Gabber mobile app supports users in collecting spoken audio data, with participants responding to pre-defined prompts written by the research coordinators. Participants can either record themselves or others responding to these topic prompts, supporting qualitative data collection which is distributed, large-scale, low-cost and participatory. With the interview participants' explicit consent, audio recordings are then made available on the Gabber website for others who have been involved with the project to listen to, highlight, tag with themes and comment upon. In this way, stakeholders partaking in public engagements are able to contribute throughout the entire process, and are given more transparency around how their data is being used. As a part of a different research project, I helped produce a fork of Gabber, called TalkFutures \citep{Rainey2020}. TalkFutures was developed as a component of Strategy 2030, a research project within the International Federation of the Red Cross and Red Crescent which aimed to understand the issues and challenges which the Federation would face in the near future. As with Gabber, TalkFutures made it easy for participants to contribute semi-structured, qualitative audio data (e.g. in the form of interviews). Uploaded audio recordings were made available on the TalkFutures website, where they could be filtered by discussion topic or even which National Society the participants worked in. By the conclusion of the Strategy 2030 investigation, members from 86 different National Societies contributed recordings using the application.

The approach taken by Rainey et al. in both Gabber and TalkFutures is reminiscent of Fox and Le Dantec's `Community Historians' project (discussed in \ref{sec:PlaceMakingHCI}), in that stakeholders contributing to the research are treated as collaborators rather than participants. While Community Conversational encountered institutions simply engaging stakeholders as a sort of tokenistic gesture to justify previously made decisions, the collaborative structure of Gabber almost precludes that---stakeholders are able to see each others' data, contribute to its analysis and hopefully more easily understand how the researchers reach their final conclusions. These projects highlight that including stakeholders to give creative input throughout an engagement process is key to gaining the richest understanding of issues within place. Furthermore, it demonstrated how giving greater levels of control to place stakeholders (without interference from top-down institutions) can result in greater levels of transparency and perceived authenticity.

\subsubsection{Remix Portal}

As well as highlighting opinions and values within place, some Digital Civics projects have also started utilising local communities as educational resources. For example, Remix Portal is a web platform designed to connect schools with musicians within the local community through the teaching of music remixing \citep{Dodds2017}. The tool is used within school music lessons, allowing children to place effects and mix the individual instrument stems of given tracks created by local musicians. After remixing, children are given `show and tell' feedback through the web portal: feedback can be left through the website directly onto the remix, and the mix board's controls manipulated to demonstrate suggested changes to the learner. This feedback can be left by fellow students, teachers, or---most significantly---the original musicians. As a result, Remix Portal provides a platform upon which schools are able to directly connect with local experts within nearby communities. This gives students a greater appreciation of their local music scene, as well as a realisation that it was possible to create great creations without needing to have the opportunities given to the biggest stars. As one student noted: `\textit{Everyone expects it to be the big famous people that you listen to, but we've got people living nextdoor to us that are just as good.}' The students were also particularly motivated knowing that the original musicians would listen to their remixes, as they wanted to `impress them'. The experts also claimed to benefit from the study from exposure to new audiences, inspiration from the students' submissions and even the opportunity to `pay back' teachers who had previously nurtured their talent.

Remix Portal demonstrated that it's possible for learning technologies to introduce new layers of infrastructure for local knowledge sharing. By taking part and being exposed to local talent and expertise, the students were able to reflect and re-evaluate their interpretation of place within the context of music production---not only highlighting community expertise which they may have previously been unaware of, but also potentially opening new interests and hobbies for the students to further explore. Remix Portal demonstrates that learning technologies can help bridge formal education contexts with local communities of practice, resulting in students having a greater appreciation of the value of places close to them. Such findings highlight the potential for mobile learning technologies to do the same.

\subsection{Spatial Citizenship}

Outside of the Digital Civics agenda, Gryl and Jekel argue for a greater utilisation of geo-information systems (GIS) in secondary schools \citep{Gryl2012}. They argue that common reasons for including GIS within schools---such as preparing students for entering the workforce by introducing them to technical tools (which are often outdated by the time students enter industry), or that spatial thinking is a key competence for problem solving across multiple subjects (with `spatial thinking' usually being very narrow and limited, not accounting for social elements such as human intent, power and political processes)---are misguided or limiting. 

Instead, Gryl and Jekel argue that `spatial citizenship' is their preferred approach for including GIS in secondary education. Rather than configuring the use of technology to prepare students for entering the workforce or meeting scientific requirements, spatial citizenship is centred around the everyday lives of individuals. They claim that education for spatial citizenship `\textit{aims at enabling secondary students to devise alternative spatial scenarios, and to participate competitively in society with the help of GIS}.' Gryl and Jekel argue that such an education is necessary to prepare students to be active `spatial citizens': those who are able to use geographic information systems to \textit{`critically appropriate space by democratic means in order to participate in society.’} In short, it's teaching students how to use spatial data to be citizens, rather than simply workers. The goals of spatial citizenship education---and a comparison to previously existing models of citizenry---can be seen in Table \ref{tab:SpatialCitizen}. In order to fully participate in society, the authors argue that learners should be able to access, read, interpret and critically reflect on information surrounding a space, as well as express and share their own location-specific opinions. Gryl and Jekel argue that citizens' access to and understanding of data can be a society-changing factor: they posit that data can be used to exercise control over others or work towards solving the world's problems, and that the absence of it can allow for such problems to be neglected or hidden---particularly with the construction of `alternative facts' \citep{gryl2018}. 

The links to the previously discussed Digital Civics projects are obvious---under this context, there is a clear scope for the concept of the spatial citizen to play a role in local knowledge sharing and exposure to existing communities (e.g. Remix Portal), collecting opinions within given areas (e.g. PosterVote), platforming stakeholders' opinions on given topics (e.g. Gabber, Community Conversational, ThoughtCloud), providing data as evidence for advocacy (e.g. Data:In Place, WheelieMap) or the creation of new GIS technologies to meet a community's specific needs (e.g. FeedFinder, AppMovement). An important aspect of civic education is giving the learner the skills and knowledge necessary for active involvement in society, through information sourcing, critical analysis and debate. Highlighting the importance of active citizenship, Walzer claims that \textit{`the passive enjoyment of citizenship requires, at least intermittently, the activist politics of citizens’} \citep{Walzer1983}. Spatial citizenship allows for the re-contextualisation of the above projects into the field of education, exploring how the development of active spatial citizens can be supported through the use of place-based technologies. Furthermore, it highlights under-explored opportunities for `civic mobile learning': where mobile learning technologies incorporate some or all of the dimensions shown in Table \ref{tab:SpatialCitizen} to encourage or introduce learners to active spatial citizenry.

{\raggedright
\begin{tabularx}{0.95\textwidth}{ p{27mm} | X X X}
    {\small\textit{Dimension of citizenship}}
    & {\small\textit{Dutiful citizen}} 
    & {\small\textit{Web 2.0 citizen}}
    & {\small\textit{Spatial citizen}}\\
    \midrule
    {\small Knowledge }
    & {\footnotesize National history emphasizing common experiences and myths; government functions}. 
    & {\footnotesize Generational histories emphasizing life experiences; finding and assessing credible sources of information outside the official domain. }
    & {\footnotesize Spatial embeddedness of social life; constructions of space and deconstruction methods of spatial information. } \\
    \hline
    {\small Organisation }
    & {\footnotesize Knowing about lobbying, parties, civic groups; reasons to join these.} 
    & {\footnotesize Role of social networking, reasons to and effects of joining social networks}.
    & {\footnotesize Geo-communities; effects of everyday application of GI; spatial privacy issues}.  \\
    \hline
    \small Communication 
    & \footnotesize Understanding conventional media. 
    & \footnotesize Participatory media skills (e.g. blogging); learning how to reach audiences with digital media. 
    & \footnotesize Participatory geo-media skills: competitive lay mapping, volunteered geographies, learning about the power of maps.  \\
    \hline
    \small Participation 
    & \footnotesize Voting, campaigning and courts of justice. 
    & \footnotesize Identification of paths to join or organize effective peer advocacy networks.
    & \footnotesize Identification of paths for spatial analysis and representation in decision-making processes.  \\
    \hline
    \small Attitudes 
    & \footnotesize Trust in government and institutions of the state. 
    & \footnotesize Empowerment, trust in networks, confidence in participatory skills. 
    & \footnotesize Habit of reflection on own and others' spatial constructions, confidence in participatory skills regarding spatial planning.  
\end{tabularx}
}
\captionof{table}{Gryl \& Jekel's spatial citizen, compared to the education of other models of citizenship such as the `Web 2.0 actualised citizen' \citep{Gryl2012}}~\label{tab:SpatialCitizen}

% I'm not hacky, you're hacky %
% I just don't fully understand LaTeX %
\setlength{\parskip}{1em}

\section{Summary}

This chapter briefly introduced the concepts of space, place and infrastructure, as well as some of the HCI research that has been undertaken to understand how technologies might influence the place-making process.

Place and space are different---albeit related---concepts: while space might describe a geographical area which our bodies can perceive, place is much more abstract, describing the meanings that individuals ascribe to physical or abstract (e.g. online) spaces. Space and place also feature layers of physical and social infrastructure, which can be interpreted on a similarly personal level and can shape how individuals experience place. For these reasons, place-based technologies need to at least be aware of the personal nature of a place, as people's experiences, relationships and interactions with it can be jarringly different.

Our relationships with places are built through experiences with them over time (`place-making'), however the nature of these experiences can change how the relationship develops. For example, a person might subconsciously experience place remotely through seeing it in movies (`vicarious insideness'), or deliberately attempt to understand a place in-depth through more thorough investigation (`empathetic insideness'). Furthermore, these representations of place may be deliberately or inadvertently sanitised through `museumisation', leading to an inauthentic, `Disneyfied' experience of place. Place-making is frequently defined as being made up of two complementary concepts: place attachment (the degree to how much someone values or identifies with a place, from it fulfilling their needs or defining them as an individual) and place meaning (the meanings that individuals ascribe to settings that they are familiar with, reflecting their environment, social interactions, culture, politics, economics and history). 

As place-making occurs through \textit{experiencing} place over time, technology is able to influence the process. While frequently derided as elements that distract from experiencing the real world, digital technologies such as smartphones also provide new place-making opportunities by opening up access to new media formats, discourse and information and places near and far. Technologies have already allowed people to share their personal experiences with and knowledge of place with others, supporting new avenues for building vicarious and empathetic insideness through the sharing of place meaning. Many Human-Computer Interaction researchers have investigated the potential roles for technology within place-making and users' interactions with place and socio-technical infrastructure. Researchers have repeatedly emphasised the importance of working closely with stakeholders, through co-design and action research methods. Working closely with the individual and the local can grant insight into personal stories and individual interpretations of place and infrastructure, minimising inauthentic representation and inappropriate design. Researchers have also noted the importance of providing value to the stakeholders they work with, and that giving stakeholders a degree of ownership over the project notably increased meaningful engagement.

Existing Digital Civics projects have supported and empowered place stakeholders with new methods for knowledge sharing, self-representation and the expression of their needs and values---all of which could be utilised in the place-making process. Gryl and Jekel's `spatial citizen' provides an opportunity to re-frame technologies designed to promote place-based citizenry to the field of education. These projects highlighted: the value of engaging with place while in authentic contexts; that digital technologies can highlight others’ experiences and interpretations of place, as well as communities of practice, as educational resources; the value of giving communities significant degrees of creative freedom and approachable tools; and a demand for technology to empower place stakeholders in self-representation and civic action without contributions first being filtered through top-down institutions. I identified an under-explored opportunity for technologies to offer `civic mobile learning': experiences within authentic learning contexts, which introduce learners to new interpretations of place, communities of practice and opportunities for active citizenry.