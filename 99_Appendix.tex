
\chapter{Documents}

\section{Information Sheets}
\label{app:infoSheets}

\textit{The following is representative of the content on information sheets given to participants. The content varied slightly with each study, according to the purposes and processes of the study, as well as if consent was needed from an adult participant or by the guardian of a participating child. The contents below have been edited to maintain anonymity.}

\vspace{10mm}

Dear Sir/Madam,  

\vspace{5mm}

We would like to invite your child or a child you care for to participate in a research project, OurPlace, being run by Open Lab, Newcastle University. 

\vspace{5mm}

\textbf{What is this study about? }

The aim of the research is to explore how digital technologies can support the use of local heritage sites and their surrounding communities as an infrastructure for learning. Encouraging outdoor learning within schools has become a high priority for Ofsted and these sites and communities are often overlooked as teaching resources. Many of these sites – such as parks – have also seen severe budget cuts over the last several years, and through projects like this we hope to increase their perceived value within local communities. 

We are developing a mobile application – OurPlace – which supports the creation and use of playful learning activities in these locations, for use by teachers, families and local community experts. 

\vspace{5mm}

\textbf{What would participation involve? }

This session will involve children and their teachers and assistants using the OurPlace mobile application in [redacted]. This will take place during a normal visit to the park, meaning no additional teaching time will be taken up by the research. 

The children will be asked to complete various activities using the app. These activities have been created by teachers from the school, and have been designed to take advantage of the park environment as a learning resource. Using the app, children may be asked to take photos, record video, draw pictures and plot locations on maps. The materials that the children create can later be used in follow-up classroom activities. The children’s usage and impressions of the application will be used to assess it and shape its development going forward. These impressions may be audio recorded and transcribed for later reference. 

Attached is a consent form, with multiple elements for you to confirm. If you are not comfortable with any of them, feel free to not tick them. The researchers will not take photos of your child without your consent, nor will they collect any information that could be used to identify them. Any images which the children take will be accessible by their teacher.  

\vspace{5mm}

\textbf{What happens if I change my mind during the study?}

It is up to you whether you want your child to take part. You can choose to withdraw your child at any time if you no longer wish for him or her to take part, even after the study has finished.  

\vspace{5mm}

\textbf{Confidentiality} 

All of the data items collected, including photos, audio recordings and transcripts, will be anonymised and stored securely. Only members of the research team will have access to them. Our findings will be published in written reports that will not identify your child or show that they have taken part. If photos taken by the researchers or by children using the application show other children, we will blur out their faces when used in publications or publicity material.  

\vspace{5mm}

We would be grateful if you could complete the attached consent form and return it to your child’s teacher prior to the school trip. 

\vspace{5mm}

Thank you for your time and consideration.   

\vspace{5mm}

Sincerely, 

Dan and Ahmed 

\vspace{5mm}

Contact Details:
Dan Richardson: d.richardson@newcastle.ac.uk 
Ahmed Kharrufa (Supervisor): ahmed.kharrufa@newcastle.ac.uk 

\newpage
\section{Consent Forms}
\label{app:consentForms}

\textit{The following is representative of the content on consent forms given to participants. The content varied slightly with each study, according to if consent was needed from an adult participant or by the guardian of a participating child.}

\vspace{5mm}

\textbf{Parents' Consent Form -- OurPlace}

\vspace{5mm}

{ \RaggedRight
\begin{tabularx}{\linewidth}{ | X | p{30mm} |} 
 \hline
  & Please tick if you agree \\ 
  \hline
 I have had the purpose of this study explained to me. &  \\ 
 \hline
 I have had the opportunity to ask questions about the project and my child’s participation. &  \\ 
 \hline
 I understand that my child does not have to take part. His/her participation is voluntary and can withdraw from this study at any time. & \\
 \hline
 I allow the researchers to take photographs of my child partaking in the study. I understand that any photographs of participants will be stored securely, and will be censored of any identifiable features if published. & \\
 \hline
 I agree to audio recordings being made of my child’s impressions of the application. I understand that these recordings will be transcribed, and any personal details will be removed. & \\
 \hline
 I agree to the use of unnamed quotes in future publications of this work and I understand that any subsequent publication of this research will not identify my child or me by name.   & \\
  \hline
I understand that I can contact the researchers at any time and I have been told how to do this. & \\
  \hline
I understand that any personal data that my child and I provide will be retained and processed by the researcher in accordance with the General Data Protection Regulation (GDPR, 2018). & \\
  \hline
I consent to my child’s participation in this study.  & \\
 \hline
\end{tabularx}
}

\vspace{5mm}

Name (Participant) \rule{5cm}{0.15mm}

Signed (Parent/Guardian) \rule{5cm}{0.15mm}

Signed (Researcher) \rule{3cm}{0.15mm} Date \rule{2cm}{0.15mm}

