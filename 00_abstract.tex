% ************************** Thesis Abstract *****************************
% Use `abstract' as an option in the document class to print only the titlepage and the abstract.
\begin{abstract}

The last decade has seen a significant reconfiguration of the UK's public services through policies of austerity. Severe funding cuts have been made to many local councils, resulting in various services---such as the maintenance of local parks---to be cut from some authorities' funding altogether, with their upkeep instead relying upon volunteerism. This has led to the stakeholders of many public places (particularly places of heritage) making efforts to preserve and highlight the locations they care about through self-organised, `DIY' approaches. In an effort to promote the sustainable preservation of the places they care for, these stakeholder groups are frequently turning to mobile technologies to engage with new audiences and promote the value of place to potential future volunteers. This has presented an opportunity to explore how local communities and the places they care for can act as layers of social infrastructure, and how such infrastructures can be harnessed by mobile technologies as learning resources.

This thesis explores this design space through a series of stakeholder engagements, resulting in implications for designing digital platforms which harness places’ existing multiple infrastructures as resources for civic learning. This thesis also contributes OurPlace: a mobile learning platform which supports stakeholders and learners in creating, sharing and engaging with place-based mobile learning activities through seamless learning experiences. Through a combination of longitudinal ethnographic studies with volunteer community groups and a mix of long and short-term case studies with schools, OurPlace is used to explore and demonstrate how mobile learning technologies can be used by place stakeholders to promote place-making, increase engagement in follow-up activities and support the leveraging of physical and social communal learning resources.

Finally, this thesis also contributes insights gained from multiple school engagements exploring the use of place and mobile learning technologies in project-based civic learning. These include how mobile technologies can harness students’ existing desire for independence, and how they can be configured to leverage the physical and social attributes of place and community as learning resources. These findings inform the design of a framework for `project-based mobile learning', along with recommendations for its re-configuration in response to contextual constraints.
 

\end{abstract}
