% ************************** Thesis Abstract *****************************
% Use `abstract' as an option in the document class to print only the titlepage and the abstract.
\begin{abstract}

The last decade has seen a significant reconfiguration of the UK's public services through policies of austerity. Severe funding cuts have been made to many local councils, resulting in various services---such as the upkeep of local parks, and educational activities within them---to be cut from some authorities' funding altogether, with their upkeep instead relying upon volunteerism or charges. Coinciding with an increase in the use of mobile technologies in schools, stakeholder groups are frequently also turning to them in an effort to promote the places they care for: attempting to engage with new audiences and promote the value of place to younger generations. 

This thesis explores the design space for mobile learning platforms which harness places and communities as resources for both formal and informal learning, and how such technologies can be used by stakeholders to share their knowledge and further their own agendas. This design space is then further explored through the design, development and evaluation of OurPlace---a mobile learning platform consisting of Android and iOS applications and a supplementary website. Through multiple engagements, OurPlace was shown to support community members, teachers and learners in creating, sharing and engaging with place-based mobile learning activities through seamless learning experiences. To further investigate how such mobile learning technologies and local resources could be effectively used within formal education, this work also proposes a framework for `project-based mobile learning', applying and evaluating this framework using OurPlace in three different schools and a summer school of Travelling Showchildren, working within the unique constraints of each.

Through a design-based research approach, this project combines findings of longitudinal observational studies with volunteer community groups and a mix of long and short-term case studies with schools to contribute: implications for designing digital platforms which harness places’ existing social infrastructures as resources for civic learning; OurPlace, a platform designed to harness these resources; and the introduction and demonstration of a generalisable framework for structuring the use of such mobile learning technologies within project-based learning, along with recommendations for its re-configuration in response to contextual constraints.

\end{abstract}
