\chapter{Teacher-Led OurPlace Engagements}
\label{chap:Teachers}

\section{Overview}

As well as investigating how mobile learning platforms such as OurPlace could be used by community experts and stakeholders for sharing their knowledge and values, we were also interested in exploring if and how such tools could work within a formal education context. To this end, at the same time as working with the community heritage groups we were also working with teachers to investigate the use of ParkLearn as a seamless, place-based learning tool, garnering feedback as to how the application could be improved. This involved a longitudinal study with a local primary school, as well as several one-off studies with other schools in the area. The work covered by Section \ref{sec:LongitudinalSchool} was part of a body of work which was peer-reviewed and published at MobileHCI 2018 \citep{Richardson2018}, with the paper being co-authored by Doctors Pradthana Jarusriboonchai, Kyle Montague and Ahmed Kharrufa.

As these studies were running concurrently with the community engagements covered in Chapter \ref{chap:Community}, the application still existed as `ParkLearn', without the features and re-branding of `OurPlace'.

\section{A Longitudinal Study with a Primary School}
\label{sec:LongitudinalSchool}

I was put into contact with a teacher at a local primary school through a co-researcher, Dr Jarusriboonchai, who was working with them on a different research project. Aiming to to evaluate and further develop the ParkLearn application, I worked with this teacher (Teacher 1, Year 4---aged 6-7) for a period of roughly one year, using ParkLearn multiple times on school trips, in the classroom and on the school's grounds. I also worked with another teacher (Teacher 2, Year 6---aged 10-11) in the same school, although this was limited due to examination pressures. The longer study period was chosen for two main reasons: to mitigate the influence of `novelty' in the children’s engagement with the technology (the hope was that once students had used the application multiple times, more authentic engagement could be observed---rather than simply excitement at the chance to use new technology) \citep{Sharples2013} and to see how the teachers' approaches to Activity creation would change as they gained experience with the application over time. To this end, rather than have the research team create Activities for the students complete, the teachers used the application at their own volition: they took on a co-researcher role, creating the ParkLearn Activities independently and developing their own design ideas. The teachers chose to use the application a total of eleven times between the two classes during the study period: twice with the Year 6 class, and nine times with Year 2.

This section will describe the school as a research context; how we introduced ParkLearn to the teachers; how the teachers introduced the app to the students; the Activities that the teachers created; and observations and discussion of the application's use in this context.

\subsection{Context}
The school is situated in one of the most economically deprived areas of England: the ward in which the school sits features the highest crime rate within the constituency, 26\% of its population are within the 10\% most deprived in the UK, and 15\% of the children within the ward live in poverty (down from 27\% in 2010). The area's life expectancy is 73 years for people born male, and 78 for people born female---significantly lower than the national average of around 88 years.

Teacher 1 claimed that the school had particular difficulty in engaging with many of the students' parents, many of whom were unlikely to appear at parents' evenings or other similar school events (for example, only two parents attended a meeting about a class's transition between school Key Stages). Despite this, the school itself is of a very high standard, having been awarded an `Outstanding' rating in their latest Ofsted (the UK’s Office for Standards in Education) review. While they don't have access to their own transportation (having to instead hire coaches), the school lies within a short driving distance of the park referred to in Section \ref{sec:TalkingStatue}. It also features its own grounds, including a tarmacked playground, sizable green space and a small wooded area.

As this was a primary school, teachers each teach individual classes for all subjects, including technology-related subjects. While Teacher 1 was not very confident in using digital technology, Teacher 2 was seen as the school's `whizz-kid' teacher. As they taught a slightly older class of students, Teacher 2 frequently tasked them with doing online research during classroom activities. To this end, the school had within recent years established a partnership with Samsung, who had supplied them with a smart classroom display and 20 Android tablets, which were a shared resource amongst all the school’s classes. While a shared resource, the older classes were given priority, and Teacher 2's class regularly used them during classroom activities. However, as there were typically fewer tablets than there were children per class (typically ~30), tablets were often shared between pairs of students.

\subsection{Introducing the Application}

Prior to the application being used in class, we sat down with the teachers for an hour to give them a brief overview of the study, the application itself and how we had imagined it could be used. To serve as examples for what the app could be used for, we had created two simple Activities beforehand. The first Activity took the learner on a bug hunt, using primarily camera-related Task Types to find and photograph insects in an outdoor environment. The second Activity was a more creative one about movie making, and involved the learner creating materials for their own film (such as recording `Foley' sound effects, designing a poster and recording videos of specific shots). While Teacher 2 understood the ParkLearn application very quickly and didn't feel the need to engage with it very much, Teacher 1 took longer to be comfortable with it due to their lack of confidence around digital technology. However, it wasn't long before Teacher 1 also understood the app, and they were enthusiastic about using it independently: after going through the example materials, they created their own short Activity containing a \textit{Location Hunt} and several \textit{Match Photos}, going outside onto the school grounds to take the target photos and try out the created Activity. 

We let the teachers decide how they would like to introduce the students to the app, and they decided to create a simple Activity together for use in introductory lessons with their respective classes. Their first Activity was very exploratory, designed for use in the classrooms to see how easily the children could use the application (Table \ref{tab:TeacherActivities}, row 2). This Activity focused mainly on camera-related Task Types, as the teachers perceived them to be more immediately understandable interactions and could be easily completed in the classroom environment. In contrast, while they were excited by \textit{Location Hunt}, the teachers were worried that the distance measurement would be too abstract for some of the children (particularly the younger ones), and it would require the introductory session to be held outside. This session was as much for the teachers to get used to using the application as it was for the students---the teachers were able to practice instructing the students on how to open a particular Activity (opting to use the share code, rather than display a QR code), and see how the students' responses could be uploaded and viewed after the session.

The introductory sessions with the two classes went well, and by the end of them the children largely understood the ParkLearn app. For the Year 6 children, this may have been because they were already extremely comfortable with using the tablets and familiar with standard Android application interfaces, and so had very few issues understanding the application’s design language. However, some of the younger Year 4 children were less able readers, and so struggled to understand even the simple instructions for each Task created by the teacher. To mitigate this, subsequent versions of the ParkLearn app featured the text-to-speech function, which read aloud the Task’s instruction at the push of a button available on each Task's card.

\begin{table}[]
    \centering
    \begin{tabularx}{\linewidth}{ 
| p{4mm} 
| >{\raggedright\arraybackslash}X 
| >{\raggedright\arraybackslash}X 
| p{13mm}
| >{\raggedright\arraybackslash}X 
| >{\raggedright\arraybackslash}X 
|}
\hline
\small\textit{\#}
    & \small\textit{Activity Title} 
    & \small\textit{Used Task Types}
    & \small\textit{Uploads}
    & \small\textit{Uploads' Cumulative Contents}
    & \small\textit{Notes}\\
\hline
\small 1 
    & \footnotesize `Our School Grounds' 
    & \footnotesize 2 Record Audio; 1 Take Photos; 1 Record Video; 1 Photo Match 
    & \footnotesize 1  
    & \footnotesize 2 audio recordings; 2 photos; 1 video
    & \footnotesize Only used by Teacher 1 to test the application\\
\hline
\small 2
    & \footnotesize `Learning to use ParkLearn' 
    & \footnotesize 3 Take Photos; 1 Draw on Photo; 1 Record Audio; 1 Record Video 
    & \footnotesize 29  
    & \footnotesize 91 photos; 29 drawings; 29 audio recordings; 29 videos
    & \footnotesize Used in the classroom by both teachers to introduce the children to ParkLearn\\
\hline
\small 3
    & \footnotesize `Trip to X Hall and Gardens' 
    & \footnotesize 2 Take Photos; 2 Photo Match; 1 Record Audio; 1 Record Video; 1 Map Marking; 1 Location Hunt 
    & \footnotesize 8  
    & \footnotesize 43 Photos; 8 audio recordings; 8 videos
    & \footnotesize Tablets shared in pairs\\
\hline
\small 4 
    & \footnotesize `X Park---Statues and Monuments' 
    & \footnotesize 4 Take Photos; 2 Record Audio; 1 Record Video
    & \footnotesize 0  
    & \footnotesize -
    & \footnotesize Responses weren't uploaded\\
\hline
    \small 5 
    & \footnotesize `Exploring X Park's Flower Garden' 
    & \footnotesize 5 Photo Match; 2 Take Photos; 1 Record Video
    & \footnotesize 5 
    & \footnotesize 46 photos; 5 videos
    & \footnotesize Some submissions were lost as tablets were re-used prior to upload\\
\hline
    \small 6 
    & \footnotesize `X Park---First Visit' 
    & \footnotesize 9 Photo Match; 1 Take Photos; 1 Record Video; 1 Record Audio
    & \footnotesize 7 
    & \footnotesize 78 photos; 7 videos; 7 audio recordings
    & \footnotesize Some submissions were lost due to software bug\\
\hline
    \small 7 
    & \footnotesize `KS1 Tree Day' 
    & \footnotesize 5 Photo Match; 1 Record Audio; 1 Text Entry; 1 Take Photos
    & \footnotesize 15 
    & \footnotesize 67 photos; 15 audio recordings
    & \footnotesize Children asked to enter their names in the Text Entry Task\\
\hline
    \small 8 
    & \footnotesize `Zoological Gardens' 
    & \footnotesize 6 Take Photos; 1 Record Video; 1 Record Audio
    & \footnotesize 12 
    & \footnotesize 173 photos; 12 videos; 12 audio recordings
    & \footnotesize n/a\\
\hline
    \small 9 
    & \footnotesize `Welcome to Class 2' 
    & \footnotesize 2 Record Video; 1 Take Photos; 1 Record Audio
    & \footnotesize 4 
    & \footnotesize 8 videos; 7 photos; 4 audio recordings
    & \footnotesize n/a\\
\hline
    \small 10 
    & \footnotesize `Year 2 at X Keep' 
    & \footnotesize 1 Record Video; 1 Take Photos;
    & \footnotesize 5 
    & \footnotesize 5 videos; 13 photos;
    & \footnotesize One of the responses was from the site's manager\\
    \hline
    \small 11 
    & \footnotesize `X Zoo 2018' 
    & \footnotesize 1 Record Video; 1 Take Photos;
    & \footnotesize 13 
    & \footnotesize 10 videos; 98 photos;
    & \footnotesize n/a\\
\hline
\end{tabularx}
    \caption[The Activities created by teachers 1 and 2, and the uploaded responses created by students.]{The Activities created by teachers 1 and 2, and the uploaded responses created by students. Rows 1 and 4-11 were created by Teacher 1, row 3 by Teacher 2, and row 2 by both.}~\label{tab:TeacherActivities}
\end{table}

\subsection{Year 6 Activities}

The Year 6 (aged 10-11, we engaged N=16 children) group used the application on a trip to a site popular with school groups thanks to its historical, natural and scientific features. The site featured an indoor museum and a large outdoor property featuring woodlands and ornamental gardens. As the location was a significant distance away from the school, the teacher was unable to visit the site to create the Activity in-situ. Instead, Teacher 2 prepared the Activity (Table \ref{tab:TeacherActivities}, row 3) independently on their own device the night before, designing it using online resources in combination with their prior knowledge of the location. For Task Types which required additional resources, the teacher downloaded them from the Internet rather than collecting them personally (e.g. using photographs downloaded from Google Images for \textit{Photo Match} Tasks). Using the Activity's share code, the teacher asked the students to pre-load it in the classroom while an Internet connection was still available, prior to the class leaving. Students shared the tablets one between two, with 8 tablets being used (the class was split into two groups of 15-16, with only one group using the app). 

The Activity's Tasks included: \textit{Take Photos} of the various wooden bridges present; \textit{Photo Match} Tasks of a modern water pump and an iron bridge; a \textit{Record Audio} of the natural sounds of the forest; a \textit{Record Video} of an Archimedes screw rotating; a \textit{Map Marking} Task to plot where the site's powerhouse was; a \textit{Location Hunt} to navigate the children to a mystery location (an old waterwheel); and a final Task which challenged the children to compete and \textit{Take Photos} of the most beautiful flower they could find. Unfortunately, the teacher's \textit{Map Marking} Task didn't work during the trip, due to the Task Type's reliance on Internet connectivity to load Google Maps.

Unfortunately, due to the class entering into an examination period, this was the only time that Teacher 2 was available to use ParkLearn during the study.

\subsection{Year 2 Activities}

After the introduction session, the Year 2 class (aged 6-7, N=29) used ParkLearn on three separate school trips, as well as during multiple activities on the school grounds. For the class's first trip to the park, Teacher 1 created two different Activities at home, independently on their own device. The first focused on the historical monuments and memorials in the park, and asked the children to record videos of each other explaining what each monument was dedicated to (Table \ref{tab:TeacherActivities}, row 4). The second Activity used \textit{Photo Match} Tasks to find and photograph specific flowers in the park, with a final \textit{Take a Photo} Task asking them to choose their favourite (Table \ref{tab:TeacherActivities}, row 5). These Activities largely focused on camera-based interactions, as Teacher 2 believed that the younger Year 2 children would be able to more easily understand them than the more complicated Task Types. The students didn't find the first Activity very engaging, and the teacher and assistants resorted to telling the children what to say when recording the videos. Unfortunately, many of the children's responses to these Activities were lost: Teacher 1 didn't see much value in the responses to the first Activity, and so didn't upload them. The students engaged more with the second Activity, however much of the data was wiped by students re-using tablets (this early version of the app didn't wipe Activities after finishing, ready for re-use---that feature was added in response to this).   

Teacher 1 also used the application on another trip to different local park, where the park ranger had invited their class to make suggestions as to how the it should be improved. Independent of the research team (we only found out after the trip had taken place), the teacher made another Activity which asked the students to take photos of different areas of the park, and make audio recordings which would then be shared with the park ranger (Table \ref{tab:TeacherActivities}, row 6). Unfortunately, a bug in this version of the app resulted in a loss of several children’s work, meaning that these children’s feedback was sent to the ranger as part of a classroom writing exercise instead. Teacher 1 resorted to sharing their Google account details in order to share the children's uploads with the ranger. In response, I added the ability to share `magic links' to uploaded responses on the website, which didn't require the recipient to log into the platform (as described in Section \ref{sec:ImplementationWeb}).

The Year 2 class also used the application during the school's `Key Stage 1 Tree Day', using it to identify and talk about the trees they could find on the school's ground. Teacher 1 created an Activity which included a variety of Tasks, from \textit{Photo Match} Tasks which asked students to find particular tree types, to a \textit{Record Audio} which asked the children to describe how it felt to be around nature (Table \ref{tab:TeacherActivities}, row 7). By this point Teacher 1 was getting familiar with how the app worked, and had realised that because all of the students' devices were logged into the same account, identifying who uploaded what could be difficult. As a solution, the teacher included a \textit{Text Entry} Task into this Activity, and made sure that the children entered their names into it before starting the Activity. Following this, I added the ability for Activity creators to require a name entry field be completed prior to respondents uploading results, the contents of which would be visible on the `Your Uploads' and `Responses' pages on the website (as described in Section \ref{sec:ImplementationWeb}).

\begin{figure*}
  \centering
  \includegraphics[width=0.7\columnwidth]{images/chapter07/zoo.jpg}
  \caption[Children using ParkLearn during a class trip to the zoo]{Children using ParkLearn to classify zoo animals during a class trip (using Teacher 1's Activity detailed in Table \ref{tab:TeacherActivities}, row 8)}~\label{fig:ParkLearnZoo}
\end{figure*}

The teacher also used the application on a class trip to the zoo (Figure \ref{fig:ParkLearnZoo}), with a simple Activity consisting of Tasks which asked the children to take photos of animals, categorizing them into different types. The Activity then asked the students to record video clips, asking them to present a fact of their choice that they had discovered during the day and found particularly interesting (Table \ref{tab:TeacherActivities}, row 8).

The class' final use of the app took place at the end of the school year. Teacher 1 created an Activity which asked four children from the class to give advice to the younger Year 1 students about being in Year 2 (Table \ref{tab:TeacherActivities}, row 9). The Activity asked the children to choose and photograph an area of the classroom, and record a video giving advice about the do’s and don’ts. The teacher chose these four children either because they would benefit from the practice (due to a lack of confidence or, in the case of ‘Child 1’, a speech impediment), or because they were especially enthusiastic about using ParkLearn again. Once the children's responses were uploaded, the teacher played their videos to the class on the classroom's projector via the ParkLearn website.

Teacher 1 went on to use the application twice with the new Year 2 class the following academic year, by which time the app had been re-branded to OurPlace. The first time this new class used the app was during a trip to a medieval castle, for which the teacher had prepared a very simple Activity consisting of only two Tasks: a \textit{Take Photos} and a \textit{Record Video} (Table \ref{tab:TeacherActivities}, row 10). Due to the space limitations of the castle's cramped rooms, the class was split into two groups---one of which were given tablets with OurPlace loaded onto them. Unfortunately, the tablets weren't used much, due to the students having hands-on activities and tours with the castle staff (Teacher 1 didn't want the students to disrupt the tour through being distracted by the tablets). As a result, not many responses were uploaded by the students. Over a year later, however, one of the castle staff found the Activity (it had been tagged as being at the castle, and so was discoverable by location) and recorded his own responses to Teacher 1's two Tasks.

The final Activity created by Teacher 1 was for the new class's trip to the same zoo as the previous year . While the previous Activity for that location was somewhat structured (with multiple goals for \textit{Take Photos} Tasks), the teacher instead decided to make the new Activity as open as possible, simply asking the students to take photos and videos of things which interested them (Table \ref{tab:TeacherActivities}, row 11). Again, the class was split into two groups, with only one half having access the the OurPlace app. The number of photos uploaded averaged around 8 per student, less than the previous year's Activity, which averaged around 14 photos.

\subsection{Observations of the Teachers' Use of OurPlace}

For the ease of presentation, the observation and interview data from this longitudinal study has been structured into the three themes that emerged from the thematic analysis of the engagements held with the school: \textit{Supporting Seamless Learning Practices}; \textit{Engagement and Empowerment Through Ownership}; and \textit{Supporting Civic Engagement and Inquiry}. This section will cover the observations and data relating to these themes, for discussion in Section \ref{sec:TeacherDiscussion}.

\begin{figure}
    \centering
    \begin{tikzpicture}
    \pie[explode=0.2, text=pin, radius=3.5]{
    12/Record Audio,
    34/Take Photos,
    2/Draw on Photo,
    16/Record Video,
    34/Photo Match,
    2/Text Entry}
    \end{tikzpicture}
    \caption[Teacher 1's usage of Task Types across their created Activities]{Teacher 1's usage of Task Types across their created Activities, including the Activity made alongside Teacher 2.}
    \label{fig:TaskTypeUsage}
\end{figure}

\subsubsection{Supporting Seamless Learning Practices}

Children in both age groups were easily able to use the application to independently collect data, allowing them to make the most of being in the field. The application's ability to support children responding through multimedia (images, video, audio) allowed for them to immediately collect data and record their reflections on it, without struggling with poor writing skills and virtual keyboards. This was especially true with the younger children, many of whom weren't strong writers (especially when tasked with writing on virtual keyboards). When Teacher 1 was asked about why they chose certain Task Types during a semi-structured interview midway through the study, they revealed that they had purposefully chosen ones which wouldn’t be technically challenging, allowing children to focus on the Tasks' content rather than struggling with interacting with the technology itself: 

\begin{displayquote}
"It’s automatic. They can just speak. [...] When I designed the [‘\textit{Zoological Gardens}’] activity, I basically did the video, because I wanted them not to have to write."
\end{displayquote}

The Year 2 teacher particularly favoured use of the camera, taking up over 80\% of their created Tasks (Figure 3). The children’s interactions with the technology became more purposeful as the study progressed, unhindered by a lack of familiarity and the earlier versions’ bugs. This was shown in the trip to the zoo: the children were careful to correctly classify each animal into the correct Action, trying to take as good a photograph as possible and deliberately deleting and re-taking any shots that didn’t meet their increasingly high standards. One pair of children even decided to reshoot their video recording twice to ensure their delivery of information was perfect. Despite this perfectionism, each pair still uploaded over 14 photos on average in addition to their audio and video recordings (Table 2, row 8). The older children also responded well to their Activity, particularly enjoying competing to take the best flower photo, the ‘Location Hunt’ Action’s sound and animation and competing to take the most accurate ‘Photo Match’.

The final Year 2 Activity proved to be a very different use-case for the application: most of the learning process took place independent of the technology, as the children created their presentations and practiced with each other using whiteboards and markers. The application was used to prompt the children for talking points and to record their final output. The ability to prepare and redo a video presentation proved very effective for the children such as Child 1, who would have normally struggled due to a lack of confidence. Reflecting on the activity afterwards, the teacher stated that not only did the children enjoy recording the videos, but that they also took pride in the final results: 

\begin{displayquote}
"[Child 1 would present his work], but he doesn't know what he’s going to say, he gets tongue-tied. The pride he’ll take in actually being able to give a coherent message and seeing himself back... They far more enjoyed what they were saying and what they were doing."
\end{displayquote}

While two of the children didn’t want to play back their recordings for themselves, all four participating children were eager to show their videos to the rest of the class. The other children reacted with excitement at seeing their classmates on the screen, with Child 1 even receiving high-fives. 

The teachers saw great value in how simple the app made creating a structured learning activity and collecting the children’s responses to it. When interviewed towards the end of the study, Teacher 2 noted:

\begin{displayquote}
"It’s powerful, really powerful. The way that packages it up at the end, and how immediate it is, is fantastic for me. I got it straight away, it wasn't a difficult process to do."
\end{displayquote}

Prior to the study, the school’s teachers had been manually transferring the children’s created photos and videos over USB on a weekly basis, uploading the children’s media to Earwig—an evidence portfolio suite used by the school. ParkLearn’s small output file sizes and upload system was far simpler and better suited to the teachers’ workflow:

\begin{displayquote}
"That simplicity takes away a lot of hassle—if I was to take photographs on my [tablet], I've got to get the lead, plug it into my hard drive, transfer the photos across, choose where I want to save them... Whereas this packages everything together."
\end{displayquote}

This one-step system took a fraction of the time compared to the old backup routine, meaning children’s creations would be discarded less often. Its simplicity even allowed the teachers to delegate uploading to the children. The Year 2 teacher valued that the submissions appeared on the website in the same format as they appeared on the learner’s device: 

\begin{displayquote}
"What I like about the app is you can pull together different ways of collecting and showing information. Simply by pressing that upload button, it puts it onto my screen to save and to use in that format. That’s the beauty of it."
\end{displayquote}


Because the application supported both open-ended and structured learning activities and was non-intrusive in her workflow, she plans to use the application regularly with her next class of students. When asked which school trips would benefit from this data collection, she responded: "\textit{Every trip.}"

\subsubsection{Engagement and Empowerment Through Ownership}

Both teachers created Activities which ranged between being highly prescribed (e.g. A ‘Photo Match’ with ‘Find a birch tree’) and open-ended (e.g. A ‘Take Photos’ with ‘Find what you think is the most beautiful flower’). The Year 2 class’s first Activity proved to be very prescriptive, with the teacher simply having the children repeat her words on video. While the children enjoyed recording each other with the tablets, they weren’t very engaged with the actual educational content (suggesting a high influence of the technology’s novelty factor). The Year 2 teacher noted in an interview that in these cases, the children were only really interested in viewing their own work: 

\begin{displayquote}
"When they come back after visits where we've all done the same, children's enthusiasm is not really there for what others have done. The enthusiasm is, `Can I see what I’ve done?'"
\end{displayquote}

The children took pride in the photos they had taken, showing off their creations to each other and the adults present. She started planning future Activities which would involve the children having their own topics in small groups: “They’re given a specific task and they take ownership of it, knowing that other groups are not doing that. […] When we come back to school and we feedback, there’s a great interest in what each other has produced because we’re informing everybody.” She argued that covering their own topics would lead to the children becoming experts on it amongst their friends, with the ownership of the task and knowledge empowering them through the ability to teach others: “You can kind of empower yourself through your knowledge and how you’re going to present it, and then go off and do it.”

\subsubsection{Supporting Civic Engagement and Inquiry}

The platform was used as a tool to facilitate civic participation during the class trip to the second park (Table 2, row 6), with students providing feedback and suggestions through the application. The Year 2 teacher noted that the school was struggling to engage parents in the children’s contributions: 

\begin{displayquote}
"We don’t have a lot of parental support, but, where we do, we’re looking to make cultural links."
\end{displayquote}

After this trip to the park, she had tried to make the parents aware of their children’s work through the school newsletter. Their currently used technology, `Earwig', was impractical for this:

\begin{displayquote}
"It takes a very long time to upload a video within Earwig, so we've stopped doing it, really."
\end{displayquote}

Teacher 1 suggested that mobile technologies may be better suited to highlighting the children’s civic engagements: 

\begin{displayquote}
"Our parents might go along and just say, `There's nothing there', because they don’t see the resource. However, there could be something on the app like, `We're involved in it, so go along and see what your children have done.'" 
\end{displayquote}

Other activities that she suggested could be highlighted included visits to local care homes: 

\begin{displayquote}
"Let’s say, Christmas you go to the care home. We can use ParkLearn to record what we did and use that within school, upload it to our website to share it with parents."
\end{displayquote}

This highlighting could extend to cross-cultural learning engagements: 

\begin{displayquote}
"We have a link with a school in India. The app is a perfect way of interacting with them, showing each other."
\end{displayquote}

The platform was also used to promote self-reflection on the learner’s relationship with the space. For example, one Activity asked children to record audio in response to the prompt:

\begin{displayquote}
`Why are trees special? Listen to the sound of the leaves rustling, stand amongst them and look up – how does it make you feel? Share your thoughts with us.' 
\end{displayquote}

\subsection{Discussion}
\label{sec:TeacherDiscussion}
These studies provided us with discussion points which we believe should bear consideration in future designs:

\subsubsection{Supporting Seamless Learning Practices}

The features and open nature of the application’s authorship process and website component means that it arguably supports all ten of Wong and Looi’s dimensions of mobile-assisted seamless learning [43].  This includes the four research and design gaps which they identified: use of multiple device types in different contexts (e.g. tablets and projector in the field and classroom), switching between multiple learning tasks (e.g. through combining Action Types, promoting different interactions and considerations on the part of the learner), knowledge synthesis (e.g. potential for children to create peer-learning Activities based around their own independent or group research) and the encompassing of multiple pedagogical or learning activity models (e.g. moving from individual work with tablets in an authentic context to collaborative classroom discussion around the uploaded responses). ParkLearn fulfilled DG2 by incorporating these dimensions of seamless learning, allowing it to be flexible enough for teachers to incorporate different devices, contexts and pedagogical approaches into their activities as they see fit.
  
Over the course of the study, the role of the application changed from being the learning objective to becoming a teaching support tool. In the Year 2 class’s first Activity, the technology took centre stage and became the learning focus. This overbearing design meant that not only did the children have little agency in their output, but they weren’t paying much attention to the learning environment. As suggested by Richardson et al., mobile learning design should aim to strike a balance between direct and technology-mediated environmental interactions if it is to take advantage of that environment as a learning resource [29]. The teacher’s later Activity designs sought to strike that balance, preferring the Action Types which focussed on the learning environment (Figure 3). The technology’s ‘novelty’ diminishing over time (a motivation for having the study taking place over several months [32]) also led to fewer distractions from the environment. The hundreds of photos, videos and audio recordings created and uploaded by the children during their trips (Table 2) suggest that the children were easily able to use the application to support their creative output, implying that it was successful at implementing DG5 and DG4.

By supporting the offline caching of teachers’ Activities and children’s responses on devices shared between several students, the application supported structured outdoor activities without the need for Internet access or a one-to-one device-student ratio (DG6). The technology also helped the Year 2 teacher utilise the children’s existing work for new educational activities in the classroom: using the ParkLearn website on her laptop and classroom smartboard projector facilitated full class discussion of students’ work uploaded from the tablets. Presenting the students’ responses on the website in a similar format to how they’re displayed in the application (complete with the teacher’s prompts, images and the app’s iconography) had two main advantages: it allowed the teacher to review the children’s work in the context in which it was first presented to them, and it also gave the students a familiar reference point to support them in doing related work in a different environmental context. Land’s argument that the use of visual elements can allow users of varying abilities to partake in mobile learning activities [21] suggests that young children would have struggled with the equivalent text-based, CSV style table interface on WildKnowledge [42]. Through simple interfaces which ground the learner’s context (DG4), ParkLearn supported transitioning between devices, learning environments and related activities (DG2). \subsubsection{Engagement and Empowerment Through Ownership} Throughout the study, the students, teachers and volunteers valued having ownership of their work. For example, the children took pride in their creations and showed off them to anyone that would listen. They recaptured videos if their narration could be improved; they deleted and reshot photographs if the framing wasn’t up to their own standards. As noted by Teacher 2, this pride was evident on return to the classroom where they were eager to revisit their creations. However, this enthusiasm wasn’t there for viewing other children’s responses to the same Activity. The teacher believed that ownership of the task was an important contributing factor to the children’s enthusiasm. Her plans for ParkLearn activities with her next class would involve groups all researching different topics: she argued that this unique knowledge would lead to the children becoming experts on their given subject, with the ownership of the task and knowledge empowering them through the ability to teach their peers. A natural progression of this would be for children to create ParkLearn activities for each other, moving towards giving the students greater control and supporting deeper reflection through content construction [14,17]. Success of this approach can be seen in Mobilogue, where students’ ownership of their created quizzes prompted greater engagement [16]. By supporting such different lesson structures ParkLearn successfully implements DG3.

Previous work has considered how an individual’s agency (a component of empowerment [18]) can be supported in the new Internet of Things world by applying lessons learned from the DIY movement to new, digital technologies [7]. We believe that parallels can be drawn to this with our teachers and park volunteers: they were able to fulfil their goal through technology, (mostly) independent of the usual top-down institutional restrictions which would have affected their creative control and output (DG1). Uphoff argues that an empowerment process needs to provide access to ‘power resources’—the assets which create possibilities for achieving objectives [38]. For our teachers and volunteers, Activity authorship was a power resource: it allowed them to create their own content as they saw fit and release it in their own timeframe, with minimal top-down assistance. We suggest that future m-learning designs should consider how they can empower the user through content ownership. In ParkLearn, this was achieved by granting more creative control to users and elevating them from consumers to producers of educational content.

\subsubsection{Supporting Civic Engagement and Inquiry}

The technology acted as a medium which facilitated civic participation, showing an opportunity for m-learning technologies to act as ‘gateways’ to active engagement with civic space or communities. This supports Richardson et al.’s suggestion that m-learning can engage with spaces’ social infrastructures as resources for civic learning [29]. Teacher 2 argued that an opportunity existed for technology to highlight to the parents the value of the community resources and the children’s impact on them as active stakeholders. As shown in the example of the care home visit, this highlighting could also be used to learn about the lives of members within communities who have been ostracised, forgotten or underappreciated. Through supporting multimedia data collection and sharing through multiple device types, seamless m-learning technologies can facilitate the sharing of civic knowledge and values with a wider community. While ‘Earwig’ had been impractical due to the lengthy upload process, ParkLearn’s immediacy could support such interactions without disrupting teachers’ workflow. Teacher 2 also noted that beyond simply including the children’s parents, this could also be extended to sharing values and practices in cross-cultural learning engagements (DG1). As previous work has shown, multimedia data collected through mobile devices can be used as effective cross-cultural learning resources [31]. However, opportunities exist to explore how m-learning technologies can support civic inquiry. When combined with the nQuire-it platform, Sense-it supported ‘citizen inquiry learning’ by acting as a scientific toolkit [33]. We propose that mobile technologies could also act as toolkits to support ‘civic inquiry learning’: fostering cross-cultural communities of inquiry, through the design of creative learning activities to share and enquire about civic values and practices.

\section{Limitations and Going Forward}

This study was partially limited by the time limitations placed upon our participating teachers. The application did not see as much usage by the Year 6 class due to a more demanding curriculum (resulting in fewer field trips) and the beginning of their exam season. Future work will further investigate the app’s use with this age group. Additionally, the installation of the ‘talking statue’ coincided with the end of the school term, meaning that we were unable to use it as a learning resource with the school during this study. Accessing community expertise through technology was something Teacher 2 claimed to have not considered before, but said it was “something we would use and we would access.” Future work will endeavour to investigate how community generated mobile learning resources can be used in formal education contexts. The generalizability of these studies may also be somewhat limited by the application’s park branding and imagery: the Year 6 teacher only used the application for the outdoor section of his class’s trip, opting to stow the tablets away for their indoor explorations of the museum. Similarly, it took several months for the Year 2 teacher to use the application in an activity which didn’t relate to parks, plants or animals. Future work could expand on these findings by investigating in other contexts with a context-neutral branding, which may counteract this issue.

\section{Summary}
We have presented ParkLearn—a mobile learning platform designed to support teachers and communities in creating, sharing and completing bespoke mobile learning activities. ParkLearn facilitated mobile learning in a formal education context as seamless support tool, flexible enough to support teachers in designing activities across different devices and learning contexts. Simplified processes and interfaces meant that uploading the children’s work easily fit into the teachers’ workflow, promoting follow-up classroom activities and even sharing the content in engagements between the school and the surrounding community. Through supporting creativity and independence, the platform promoted ownership of content, increasing learners’ engagement in follow-up activities. This element of independence also allowed community experts to elevate themselves to producers of rich, digital educational content—supporting them in sharing their knowledge and values with a wider community by removing the technical and financial barriers previously in place. We also identified opportunities for HCI to support cross-cultural civic inquiry, encouraging learners to share their values, knowledge and questions in a manner already embraced by citizen science research.

\section{Other Adult-Led Engagements}

\subsection{Hawthorn Primary}

\subsection{Researcher-Led Engagements}

St Teresa's

\section{Summary}