\chapter{Identifying a Design Space for Utilising Public Places as Infrastructure for Civic Mobile Learning}

This chapter gives an overview of a series of engagements with multiple stakeholders in local parks, aiming to explore the potential for mobile learning technologies to support bespoke, outdoor civic learning activities. This work investigates what it means to design for public spaces as infrastructures for civic learning. Rather than considering only parks' physical qualities or properties as resources for learning, I suggest that mobile technologies for civic learning would benefit from integrating incorporating the economic, socio-cultural and political infrastructures that comprise public spaces. The findings identify significant opportunities and challenges in designing mobile applications aimed at fostering civic learning and enhancing the development of meaningful relationships with civic space. From these findings, I draw implications for designing digital platforms which harness places’ existing multiple infrastructures as resources for civic learning. I also note the limitations of technology, and produce a generalizable model of a civic m-learning design space. The work covered by this chapter was published at Communities and Technologies 2017 \citep{Richardson2017}.

\section{Study Context}
As discussed in \ref{sec:DigitalCivics}, this project was situated within a larger socio-economic and political context of hardship currently being experienced within the UK. Significant budget cuts resulting from policies of austerity had been imposed on local government, resulting in a severe re-allocation of funds. This study concentrates on a specific consequence of this: the reduction of funding for the maintenance of local parks. Since their popularisation in the Victorian era, public parks have been a staple of British culture. Their greenery offered the working class respite from the abrasiveness of spreading urbanisation and the pollution of industry. Modern research has shown that simply having access to urban green space is essential for childhood development, as well as good mental and physical health \citep{Fiennes2015}. Today, access to green space is often limited, with 80\% of the UK’s population living in urban areas which take up only 6.8\% of its land area \citep{UKNationalEcosystemAssesment2011}. This study took place in Newcastle upon Tyne, UK, where the Newcastle City Council's Parks Service manages 12 traditional Victorian Parks, 9 countryside parks, 15 neighbourhood parks and a multitude of other sites, including several denes, reclaimed industrial sites and recreation grounds. In total, the sites managed by the Parks Service amount to over 2 million square metres of space.

Because local authorities do not have a statutory duty to fund and maintain their open spaces, local parks have had their budgets slashed under austerity measures in order to minimise the impact on other areas, such as schools and healthcare. In 2014, the Heritage Lottery fund found that 86\% of UK park managers had seen cuts to their budgets since 2010, with some local authorities considering simply selling their parks to private investors \citep{HeritageLotteryFund2014}. Between 2010 and 2019, Newcastle City Council has had to reduce its parks budget by over 90\%. In practice, this means the loss of over 80\% of the parks' full-time staff (despite park usage increasing), increasing their reliance on volunteers from local communities.  It also means that councils are being forced to explore other ways in which parks can generate income. Many authorities have reluctantly introduced---or increased---entrance fees, and are charging schools to facilitating class trips. The combination of a loss of dedicated education staff within parks, the introduction of fees to compensate for park rangers’ time, and the schools themselves having to deal with funding issues has resulted in few schools utilising the parks as learning environments. As cost-cutting measures, even fewer take advantage of the rangers’ expertise as educational resources.

The purpose of this study is not to place greater value upon local parks than the many other elements of society which have suffered from austerity measures. Instead, it serves as a case study exploring how issues which affect places as simple as parks can interact with a large number of community stakeholders in many different ways, and how learning technologies might use that as a platform for learning.

\section{Engagements}

TODO

\subsection{Formative Workshops}

We held a series of engagements over a period of eight months to understand the impacts of this context on the parks’ various educational stakeholders. These stakeholders were represented in our research by park rangers (N=5) and school teachers (N=7). The first engagements were three workshops with small groups of participants: one with just teachers, one with just rangers, and a third with both together. These workshops were made up of short activities and semi-structured interviews focussing on the participants’ relationships with parks as places, their use of the parks as learning environments, their general experiences with outdoor learning and their use of educational technologies. Visits to the parks were also held to view the educational resources that were currently available and shadow a school trip (reception class—four years old) to observe current practices.

\subsection{Application Prototyping}

We realised that to gain an understanding of children’s attitudes towards the parks and technology, something more appealing would be necessary to engage them. The initial findings from these early engagements suggested that even young children would be very comfortable using mobile technologies, and that they would better engage in activities which allowed them elements of independence and creative control. From these findings and insights gained from studying prior work, an m-learning application prototype was developed for use with the children on school trips. This allowed us to gain insights from the children in a more fun, interactive manner than the adults’ engagements.

Park:Learn (the prototype Android application) acted as a technology probe, and offered a number of modular interactions which could be configured together into outdoor learning activities (Figure 2: left). These interactions included taking a photo (‘Take a Photo’), matching an existing photo using a translucent image overlay on the camera (‘Photo Match’) recording video (‘Record Video’), recording audio (‘Record Audio’), drawing digital pictures (‘Draw a Picture’), drawing on top of taken photos (‘Draw on Photo’), marking a location on a Google Maps view (‘Map Marking’), tracking down a location by the device’s distance from a geo-coordinate (‘Location Hunt’, Figure 2: right), choosing between pre-written answers on radio buttons (‘Multiple Choice’) and simple text entry in an empty textbox (‘Text Entry’). Each of these interactions were chosen either because they put an element of creative control into the hands of the learner, took advantage of the devices’ hardware capabilities or—as in the case of Multiple Choice and Text Entry—emulated features of the learning materials currently in use. Unlike projects such as Ambient Wood and Explore! [4] (which required additional equipment or the production of 3D graphics), Park:Learn activities can be self-contained within the device and very quick to create due to the app’s modular nature. In the task model for mobile learning, these features allow activities to be designed which are intrinsically linked to the context of the park, use a wide variety of tools which allow for content construction, offer the learner a large degree of control and (for the group activities) cooperation and communication. 

\subsection{Prototype Deployments}

Two deployments were held with two groups of children in two different parks. In the first deployment (N=23, aged 4-12, recruited through an out of school club, in groups of 2-3 with a smartphone or tablet per group) students were given activities per their age: for younger children (age < 6. Figure 3, left), the app asked students to take photos of plants and wildlife, while older children (Figure 3, right) were additionally asked to Location Hunt items of historical significance in the park, record a short nature documentary style video and draw their vision of the park’s future on top of one of their own photographs (which some groups didn’t complete due to time limitations). These activities were inspired by worksheets that had previously been created by the park rangers and the discussions held with them. 
The second deployment was much more free-form in its activity design, taking place during a school group’s (N=55, aged 4-5, accessed through partnership with the schoolteacher) weekly visit to their local park. To fit into the teacher’s experiential, child-led approach for the visit, we wanted to present the application as an optional tool which children could engage with if they wished. To this end, we offered tablets running the application to 5 students (one device per child) who weren’t engaging in other activities, such as tree climbing or playing in mud. The app was loaded with free-form activities which were designed to fit the child-led learning approach, encouraging the children to catalogue their findings during their usual self-guided explorations of the allocated park area in pictures and video. Of the 5 children we approached, 3 completed the app’s activities, while 2 disengaged when they realised that it wasn’t a videogame.
Following these deployments, further workshops and interviews were held with the park rangers and teachers, with the aim to get feedback on the prototype and ideas for future developments.

\subsection{Data Collection and Analysis}

The project’s engagements were audio recorded and transcribed with participants’ consent. A thematic approach [2] to coding was performed, where codes were qualitatively analysed by the authors and then grouped into candidate themes. These themes were summarised onto paper for discussion, testing and validation before being finalised. Any quotes from participants have been anonymised.

\section{Insights}

TODO

\subsection{Self-Guided Civic Learning}

Discussions with the park rangers and teachers revealed that, in their view, outdoor learning played a critical role in children’s development as citizens. The exposure of children to new experiences, environments and community members is an essential element which helps children to discover their passions and equip them to make decisions about their future. The concept of children exploring their environments to discover and nurture new interests through independent learning is a process which was raised repeatedly during our workshop discussions:
“They pick [these professions] because they are exposed to a wider variety of natural things, they have a choice to make. […] We shouldn’t just tie our pupils into traditional classroom activities. […] Expose children so that when they grow, they can become specialists.” – Teacher 6
Our workshop participants strongly believed that this process was reliant on children’s independence—if children were to find new interests and passions to take into later life, there would have to be significant degree of autonomy and freedom of learning. 
“It’s about listening to the child and following what they want to do, as opposed to being subscribed.” – Ranger 2
While this element of self-determination was recognised as important, it was also noted that the children would still often need a teacher’s presence to act as a facilitator and an enabler for the children’s explorative curiosity:
“It is much more about allowing the children to make their choices. […] You don’t do anything apart from facilitating and listening to them.” – Teacher 4
Our participant teachers claimed that the children were discovering their passions over time through outdoor exploration, play and experimentation. Eventually, these would organically emerge into themes of personal interest which could be identified by their teachers.
“It’s about dealing with children’s own interests and passions. […] Maybe by February there’s some children who have a theme going.” – Teacher 7
We found that our initial design ideas (as with many existing mobile learning applications) were not particularly well suited to this process. Rather than allow for self-guided exploration and fluidity, our technology’s initial activity design in the first deployment had been prescriptive—meaning that the children were exploring our ideas, rather than their own. One of the more visible examples of this we saw was an activity which tasked young children to Photo Match images of types of leaves in the park. The children took this more literally than we expected, and tried to line the shot up perfectly with the leaf overlay. The result was the children cared more about taking the photograph than learning about the surrounding nature. The second deployment’s more open structure allowed for the application to take an embedded role in the session’s explorative activities—the technology became one of a selection of optional resources, including the park itself. For the participants who chose to utilise the technology, we saw that the creative potential of the application encouraged them to further engage in personal explorations of the park environment and document their discoveries (Figure 4).

\subsection{Citizenship Through Placemaking}

The teachers noted that as the children advanced through the early years of school, the focus of school activities changed from the sensory and experiential to the practical and applied. Project-based learning activities are introduced, allowing multiple school subjects to be taught around the periphery of a single class venture. In the school, an example of this was the development of the school’s garden and pond area. However, the rangers saw these projects as being opportunities for learning topics which extend beyond the current school curriculum.  They saw opportunities for civic learning, giving children an appreciation for the local parks and the work that goes into maintaining them. They wanted children to be able to explore the environment at their own pace, taking time to understand and appreciate it. Beyond this base appreciation, they hoped to instil a sense of ownership, belonging and responsibility. They wanted these learning activities to be placemaking.
“Being involved in developing [the park], studying it. So that they feel like it’s their park—not just some open space to throw cans in. […] They have ownership of it, the whole thing, and then maybe they’ll appreciate it and look after it.” – Ranger 1
For the rangers, working alongside the schools allowed them to teach children the civic value of parks. To them and the surrounding communities, the parks are more than just their physical components of open spaces, woodland and shrubbery. They have a true social value, something which needs to be treasured, nurtured and, crucially, communicated and passed-down. The rangers were very aware that the parks would soon be likely to be even more reliant on community support and volunteering. A possible route to future sustainability lies in instilling this sense of civic responsibility and duty of care. The activities designed to nurture this ownership tended to be creative in nature, allowing the children to feel like they had personally contributed to the spaces. Examples of the activities the rangers organised with schools included children creating artistic roundels to surround a new pond and designing and building a nature area. The aim was to use this newly produced area to build long lasting relationships between students and the space over the course of their academic careers: using it for experiential activities, creation and, eventually, study. 
“They’re actually involved in making the park: they planted that willow, and they planted some bulbs. So, they’ve been involved while quite young in creating this wildlife area and taking ownership of it. Hopefully, once it’s established, we can involve older kids in actually studying it.” – Ranger 1
The rangers hoped that these studies would again be mutually beneficial for both the schools and parks: as well as allowing the students opportunities for situated outdoor learning, the parks could benefit from the collected data. The students’ findings could be fed into organisations such as the Wildlife Trust and local citizen science projects, further increasing the perceived value of the parks to their surrounding communities.
There was a broad range of attitudes amongst our workshop participants concerning technology’s role in parks’ placemaking. Some were critical, viewing many technologies as distractions from the learner’s environment: the rangers and teachers alike were concerned that if a child is focussing on the technology in their hands rather than what’s surrounding them, how can they form a meaningful relationship with that space? However, there was also optimism about the use of technology as a powerful tool within this space. Some saw it as a way of furthering students’ engagement with and appreciation of the natural environment:
“I think [recording] audio would be really interesting to just listen to what the park sounds like, […] because I don’t think we listen to nature enough. […] Just appreciating it.” – Teacher 2
Other ideas included using technology as a tool through which the rich social history of the parks could be uncovered and contextualised. Through the app’s photo-matching activity, the rangers suggested that children could compare the park of today to that of a hundred years ago. These differences could be used to contextualise the changing attitudes towards the parks’ usage and upkeep, as well as foster an appreciation for the efforts of the parks’ current volunteers (a resource which wasn’t previously required, due to the large number of paid staff).

\subsection{Stakeholder Tensions}

Despite Ofsted urging schools to perform more outdoor learning activities, many teachers struggle to take their lessons outside—especially into parks. Through the workshops and interviews, we found that many aspects of the economic and institutional infrastructures surrounding the parks and schools restricted the amount of outdoor learning that could be supported, making for a difficult design space.
Recurrent and obvious was the topic of funding, for both the schools and parks alike. Most parks have had their budgets cut to the extent that they now have fewer staff; where there may have once been dedicated educational officers, rangers are having to cover in their stead in addition to their previous duties. Thus, schools are now charged for educational activities to (partially) compensate for rangers’ time, which is always in high demand.  Schools suffering from budget cuts also compound this, resulting in many choosing to stop utilising the rangers as resources for expert knowledge or even ceasing trips to parks altogether.
The nature of our society has also resulted in an unequal access to nature in many people’s lives. Indeed, many of the original Victorian parks were originally created for the health benefits of factory workers. For urban areas living with child poverty, parks are a valuable resource—both for access to nature and new social opportunities for civic learning. The theme of natural environments being social equalisers was present in our discussions: parks allow for children to exist, play and learn on a level playing field when extraneous factors are stripped away.
 “In the classroom, he’s lost. He doesn't have a TV at home, his parents are very highly educated and he finds it hard to mix in with the other children. But in the woods, it’s a level playing field, because there’s no TV, there’s no toys that match anything that they might have seen on a film or anything like that. I suppose, for him, it’s his day that he’s on a par with everybody else.” – Teacher 7
Through the discussions with teachers, additional tensions were revealed.  One was the existence of a prejudice and stigma against learning outside of the classroom:
“One [parent] complained, and said they weren’t in the learning environment. It was just this weird perception. The parents looked at it and saw ‘Look at those students relaxing, that’s not going to be a learning environment’”. – Teacher 5
Despite the teacher claiming to have ‘never had as much focus as when they were just relaxed, lying in the grass’, he found himself having to defend the practice against outside scepticism. Amongst other institutional requirements, this necessitates that teachers create schemes of work and collect evidence of learning. This target and evidence-based methodology clearly conflicts with the experiential, holistic approach used for children’s self-development. These highly structured, prescriptive formats result in little room for exploration and the unexpected. Furthermore, the targets set by the UK’s national curriculum mean that schools must teach very specific topics and meet specific targets, limiting teachers’ creative control and freedom in their activity design. One ranger (who happened to be a retired teacher) claimed that toeing the line of the national curriculum has resulted in many teachers losing the ability to teach topics in a manner tailored to students’ interests:
“You couldn’t do that now, because of the curriculum. It’s so structured. Many of the teachers have gone through that system now, and it’s hard for them to go back and think creatively about how do it – it’s been knocked out of them.” – Ranger 1
The increasingly lofty and specific learning targets for slightly older children are also affecting what is being taught in the earlier years of their education. Many schools are aiming to get children up to target earlier in their school careers—forfeiting the holistic experiences for the rote-style learning found in the later stages of school. Resistance to these top-down influences appears to be on a per-school basis:
“The curriculum is so heavy now with the grammar: our Year 6s need to know what ‘fronted adverbials’ are. […] That's so high now it’s just filtering down. The pressure on what the children need to be able to do is just increasing. And it’s our way of saying ‘we value children’s imaginations and children being children so we keep doing this’.” – Teacher 7 
However, the current institutional climate realistically only allows for these entirely freeform activities to take place during the earliest years of a child’s school life. For schools to be able to sustainably hold outdoor learning activities for older children within the existing school infrastructure, they must conform to the expectations of targets and evidence set upon them.

\section{The Social Design Space for Mobile Learning Technologies}

These engagements have shown us that civic m-learning in parks—and more broadly in civic spaces—is a rich but challenging design context. It’s clear that for a technology to be successful within these community spaces, it must be designed in consideration of the existing social ecosystems. This requires an awareness of the motivations of each place’s stakeholders and the relationships that exist between them. In our park context, a design must allow for teachers to work within a set of pre-determined parameters, with the resulting deliverables supplying evidence of learning. Similarly, rangers’ time and resources are precious due to their plethora of commitments and lack of funding, so the activity design and creation processes must be quick and easy to distribute. While teachers may aim to teach to a strict, pre-written curriculum, rangers might prefer to strengthen learners’ relationships with the park and instil a sense of ownership. The local government want the parks to remain valuable community resources, but don’t have the funding to allow the previous amount of spending to be sustainable. Technology can offer new opportunities to surface these complexities for use as civic learning resources.
Based on our findings, we present a generalizable model of the social design
space (Figure 1). In the model, we show how space and place (be they parks or
schools) comprise of multiple actors: learners, communities, institutions and
technologies. Actors refers to individuals who use the park as a space—be that
learners, teachers, rangers, volunteers or other members of the public.
Communities are multiple actors, united by a common interest, goal or issue: for
example, ‘Friends of X Park’ volunteer groups, local residents and school
groups. Institutions are those that impose requirements and/or restrictions on
the other groups: for example, Ofsted or the city council. Each of these actors
interact with the others through layers of infrastructure: for example, actors
may exist within a community of practice and the city council may introduce
financial tensions with the park rangers through policy. These infrastructures
all contribute to comprising the park as a place.
However, most current m-learning technologies only interact with the learners in physical space, oblivious of the socio-cultural, political and economic relationships that constitute place. If a technology is to be well suited for civic learning within this space, it needs to be produced with the interactions between stakeholders in mind. Civic m-learning involves more than just the learner and the space in which they reside: it also involves other stakeholders’ relationships—both with the space, and each other.

model goes here
The social design space for m-learning technologies, where relationship infrastructures connect stakeholders in space and place. ‘Traditional’ m-learning refers to m-learning technologies which don’t meaningfully engage with these infrastructures, and are either independent of the learner’s context or concentrate solely on the physical aspects of the environment.

\section{Suggestions for Designing Technologies for Civic Learning}

Based on these findings, we now present some suggestions for designing technologies for civic learning and extending the focus of m-learning technologies to include the social context.

\subsection{Create Opportunities for Giving Form to Stakeholder Values}

As suggested by Dourish and Bell, by considering the infrastructures that constitute a place/place, we can more easily understand the values that its surrounding communities associate with it. Analysis of the different actors and stakeholders at play in a space offers researchers not only a greater appreciation of the multiple practices and values of it, but also opportunities to design technologies that accommodate them and bring them in relation to one another.
An awareness of the variety and import of stakeholder viewpoints, practices and values becomes even more necessary when the communication of these values is the technology’s defined purpose. In this project, the rangers’ and teachers’ agendas were very different, despite being stakeholders in the same space and place. Understanding the contexts and spatial infrastructures (socio-cultural, institutional, financial) where these values are enacted is key to designing appropriate technologies for civic learning in these spaces. We found that despite being major users of civic spaces such as parks (and therefore are stakeholders like any other actors), children’s values, practices and views regarding parks are often overlooked. Designing for civic m-learning might entail the development of platforms that allows multiple stakeholders—including children—to express their values and practices and put them in dialogue with one another, encouraging political agency from an early age.
This potential can extend beyond the scope of individual places and communities operating within them. Indeed, m-learning technologies could operate as platforms for the sharing of values, practices and resources between and across different places and communities.  Bringing the practices and values in different communities and places into dialogue with one another can offer productive civic learning opportunities [44].  Fischer has also noted the need for collaboration amongst communities,  and claims that spatial, temporal, conceptual and technological barriers can be turned into creative opportunities [13]. Gryl and Jekel claim that the core competencies required for spatial citizenship are expression (constructing and communicating meanings of geographic information), communication (sharing those ideas and meanings with others) and negotiation (engaging in democratic discussion in an effort to find compatible meanings with others) [17]. Similarly, through activity creation, m-learning applications like Park:Learn could be used to support learners’ active citizenship through surfacing other stakeholders’ values and practices and expressing the learner’s own. Future work could investigate how m-learning technologies could assist in promoting engagement in further negotiation between stakeholders.

\subsection{Support Placemaking}

Through analysis of the workshops and interviews, it became clear that the process of independent learning and self-discovery was intrinsically linked to placemaking. Children can explore and learn about their environment at will, allowing for unique and meaningful experiences to occur. The rangers were confident that these regular and meaningful interactions over time eventually lead to the formation of relationship between the learner and their environment. Yi-Tuan claims that placemaking is made possible through individuals ‘pausing’ in space to make it place [41]. However, we argue that rather than this passive act of pausing, placemaking is promoted through doing—individuals entering an active engagement and creation process within a place and its infrastructures. To this end, outdoor learning technologies in this design space should support learners’ independent learning, curiosity and creativity. This was also seen in the second deployment of Park:Learn, where the creative potential offered by technology acted as a motivating factor. 
The teachers noted that as the children progressed in age, they transitioned from experiential and explorative activities to creative ones in which they were actively affecting their environment (and effecting change). Civic learning technologies should support this transition into active participation within society. The rangers saw this as a means of placemaking: by actively having a hand in the creation of areas of the park, children would be taking ownership and forming relationships with it. The rangers’ values where embedded into these activities, in the hope of them being passed onto a new generation. To assist in this process, mobile learning technologies might act as both creative tools and social infrastructure: empowering users to create new unique works, and share and absorb the knowledge of others in a place’s community through an ongoing dialogue and exchange between the learners and other stakeholders. As an example of how this could be implemented in an m-learning application, communities could create their own activities in Park:Learn to form their own informal curricula: sharing values, knowledge and promoting placemaking through situated learning.

\subsection{Balance the Use of Technology}

Through these extensive engagements, it appeared that stakeholders’ perceptions
of the role technology might play in parks weren’t always positive. Some of our
participants saw the inclusion of technology as something that could distract
from the learning experience and placemaking. This is a criticism which could be
levelled at projects such as [36], which shows a photograph of a class visiting
a temple, engrossed in their mobile devices rather than the environment around
them. As civic learning is tied to practices of placemaking, when designing for
civic m-learning we must be mindful not to place technology at the ‘centre’: a
technology designed for civic education and placemaking should not presume
itself to be the learning objective, and instead take a background supporting
role. We must acknowledge that there are situations where the very inclusion of
technology may not be appropriate. For example, the inclusion of a technology
could completely negate explorative outdoor learning’s equalising effects if not
all children are familiar with it. As HCI designers, we must recognise and
appreciate that the value of a physical or social space could be jeopardised by
heavy-handed outside involvement—sometimes the lack of technology in a space
could be why it is precious to begin with.

However, technologies can offer new learning opportunities which might not
otherwise be possible or feasible. For example, m-learning can give stakeholders
platforms to communicate their own values and motives concerning place; expose
the values of others to learners across time and space; augment physical reality
with digital information; and allow for dynamic and creative learning activities
thanks to the available networking and hardware features. Thus, a careful
balance must be maintained between the potential benefits of civic m-learning’s
inclusion and the risk of its overuse. A ‘sweet spot’ (specific to the learner
and the learning context) can be found in the space between completely direct,
hands-on activities without any technology use and a fully technology-mediated
approach. As the focus on one increases, the other decreases, and their
respective benefits follow (Figure 5). 

figure goes here: Balance the amount of direct and technology-mediated interactions to find the 'sweet-spot' for civic learning

\section{Summary}

In a period of increasing civil unrest and division, civic education is increasingly important.  Through insights gained from eight months of engagements with stakeholders in local parks, we identified spaces where m-learning technologies and their design processes can nurture civic education and produced suggestions for designing in these spaces. We also gained and shared an understanding of the potential placemaking role mobile technologies can play, as well as the limitations which are (or should be) placed upon them. 
We illustrated a design space which highlights the different stakeholders’ current issues and practices, drawing implications for designing platforms that support outdoor civic learning activities and placemaking. With minor adaptation, this model should be adaptable for civic m-learning in settings other than parks.

TODO